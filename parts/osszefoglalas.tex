Diplomamunkám célja egy lehetséges megoldás adása volt arra a problémára, hogy míg egyre több időt töltünk okoseszközök képernyője előtt, addig az egészséges életmódhoz szükséges tesmozgás gyakran hiányzik a mindennapjainkból.
Az alapötlet egy olyan játék elkészítése volt, ami a micropayment-ek módszereit használja fel oly módon, hogy az előnyöket nem pénzért, hanem sportteljesítményekért adja.

Ezen ötletet követve készítettem el egy Android platformon működő szerepjátékot, melynek a RunPG nevet adtam.
A játék magában hordozza a szerepjátékok jellemző stílusjegyeit, úgy mint a fejleszthető karakter, tárgyak kezelése, csaták, térképen barangolás, stb. 
A hosszú, és kihívást jelentő játékmenet biztosítása érdekében a játékbeli hősnek véletlenszerűen generált, de egyre nehezebb pályákat kell teljesítenie.
A világban utazva azonban a folyamatosan csökkenő sztamináját csak valós sportteljesítmények után tudja visszatölteni, így ösztönözve a felhasználót a testmozgásra. 
A sportteljesítmények mérését külső alkalmazások végzik, a játékhoz példaképp két meghatározó sport-adatbázis lett illesztve.

A szoftver moduláris felépítésének köszönhetően a jövőben könnyen illeszthetőek lesznek további funkcionalitások, mint például további adatbázisok illesztése, Google Play integráció, vagy PvP mód. 


