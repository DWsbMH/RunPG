%1. fejezet
Az okos-telefonok térhódítása miatt már hazánkban is a lakosság fele rendelkezik valamilyen okos eszközzel, ez a szám pedig a jövőben egyre csak növekedni fog. 
A mindennapi élet megkönnyítésére rengeteg féle alkalmazás születik napról napra. 
Egyetlen eszközön olvashatunk újságot, tudakozódhatunk a közlekedésről, vehetünk ebédet magunknak, vagy unaloműzésként játszhatunk. 
Manapság minden korosztály talál kedvére való játékot, legyen szó akár ingyenes akár fizetős verzióról. 
Napjainkban igencsak elterjedtek az olyan játékok ahol a felhasználók interakciókba léphetnek egymással. 
Ezek túlnyomó többsége az úgynevezett micro-paymentekre alapszik, ahol a játékosok csekély összegekért cserébe előnyökhöz, könnyítésekhez juthatnak. 
A „free to play, pay to win” kifejezést azokra a játékokra szokták használni, ahol az előbb említett vásárlások nélkül képtelenség megnyerni a játékot, mert túlzottan befolyásolják a játékosok fejlődését.

Egyre több ember életéből hiányzik napjainkban a rendszeres testmozgás. 
Sokan választják a séta és a bicikli helyett az autós vagy a tömegközlekedést, főleg kényelmi szempontokból. 
A technológia fejlődésével pedig egyre több olyan eszköz jön létre, amik az emberek életének kényelmesebbé tételét szolgálja. 
Ha rendelkezünk okostelefonnal, ma már a nagybevásárlást is el tudjuk intézni pár kattintással otthonról. 
Az ilyen alkalmazások célja az volt, hogy kényelmesebbé tegyék mindennapjainkat, nem azt, hogy elkényelmesítsenek minket. 
Az, hogy egyre több időt töltünk ezen eszközök előtt, csak súlyosbítja rendszeres testmozgás hiányát. 
Sajnálatos módon a felhasználók nagy része nem elégszik már meg annyival, hogy a mozgás jót tesz az egészségének, szükség lehet valamilyen fajta ösztönző módszerre. 
Ezek többféle módon is megnyilvánulhatnak, léteznek különböző virtuális díjazások, sportolásért járó pontok, amik később tárgynyereményekre válthatóak, sőt vannak tényleges pénzzel való díjazások is.

Célom a két trend összekapcsolása oly módon, hogy a játékban a gyorsabb fejlődést nem pénzkifizetéssel, hanem sportolással lehet kiváltani. 

A 2. fejezetben ismertetem az elkészülő játék hátterét, bemutatok különböző játéktípusokat. 

A 3. fejezetben ismertetem a program tervezett funkcionalitását és követelményeit. 
Illetve kitérek a játék egy fontos alapelemére az általam választott dream-iso-droid játékmotorra. 

A 4. fejezetben bemutatom az elkészült alkalmazást felhasználói szemmel. 

Az 5. fejezetben részletesen kifejtem a megvalósított játék fő alkotóelemeit, és azok implementációját. 

[maradék fejezet]