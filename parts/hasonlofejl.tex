%2. fejezet

A fejezetben részletesen bemutatásra kerülnek a fejlesztett játék elkészítését befolyásoló tényezők, és hátterük. 

%2.1
\section{A videójátékok története}
\label{videojatektortenet}

% \Picture{A társasjáték}{2/1}{width=10cm}

Videójátékoknak nevezzük azokat a típusú játékokat, ahol a játékosok egy felhasználói felületen keresztül lépnek interakcióba a játékkal. 
Története az 1950-es évekre vezethető vissza, ebben az évtizedben kezdte el foglalkoztatni az embereket az az ötlet, hogy a szórakozás élményét elektromos eszközök segítségével érjék el. 
Az első videójáték egy asztalitenisz szimulátor volt, 1958-ban készítette el William Higinbotham és a „Tennis for Two” nevet viselte. 
%tennis for two
A játék egy asztali analóg számítógépre készült el, és egy oszcilloszkópot használt a megjelenítésre. 
Az első olyan játék, amelyet már egy kontroller ősének nevezhető eszközzel irányítottak 1962-ben készült el, és a Spacewar nevet kapta. 
Itt két űrhajót irányíthatott a játékos, a másik fél letorpedózása volt. 
A Spacewar változataként jelent meg a világ első pénzbedobós játéktermi gépe a Computer Space. 
Ennek ellenére ez a játék feledésbe merült, a népszerűséget az 1972-ben megjelent PONG játék szerezte meg. 
A játék annyira népszerű volt, hogy 3 év múlva az otthoni verziója is elkészült a játéknak. 
Ezt követően megindult a játéktermi gépek térhódítása a világon. 
Az Atari mellett a Sega és a Midway cégek is belekezdtek a saját játékgépeik fejlesztésébe. 
Ekkoriban mindenki be akart szállni a játékgyártásba, emiatt gyorsan sok videójátékot termeltek ki a vállalatok. 
Az 1980-as évek elején a sok silány minőségű játék miatt halottnak nyilvánították a videójáték piacot, mert a emberek nem kívántak rossz minőségű játékokat venni, ennek következtében csökkent a keresletük. 
Az első olyan videójátékot, ami történetet mesél el a Nintendo cégnek köszönhetjük. 
A játék az 1981-ben kiadott Donkey Kong volt, ahol a ma is híres Mario karakterével kellett megmenteni a bajbajutott lányt. 
%donkey kong kep
1985-től a videójáték piac újból felemelkedett a Nintendo Entertainment System nevű otthoni játékkonzolnak köszönhetően. 
Egymás után jelennek meg különböző vállalatok játékai, ezek között található volt stratégiai, verekedős továbbá kalandozós játékok is. 
Az 1990-es évektől kezdve folyamatosan jelennek meg a felfejlesztett játékkonzolok. 
A hozzájuk készült játékok nem csak a fajtájukban térnek el, hanem különböző megjelenésükkel és irányítási rendszerükkel egyedivé varázsolják a játékélményt. 
A közeljövőben elérhetővé válik az a játékforma, amikor a játékos érzékeit becsapják a képek és a hangok, és teljesen úgy fogja érezni, mintha belecsöppent volna a játék világába. 


\section{Sportjátékok}
\label{sportjatekok}

Napjainkban nagy népszerűségnek örvendenek az olyan játékok, ahol a játékosok interakcióba tudnak lépni egymással, ez lehet együttműködés vagy egymás elleni versengés is. 
Mivel a versenyszellem velünk született tulajdonságunk, ezért a felhasználók minden tőlük telhetőt megtesznek azért, hogy a ranglista élére kerülhessenek. 
Erre alapozva sok játékkészítő nagyon kreatívan határozta meg a fejlődés árát. 
Az alábbiakban pár olyan játékot mutatok be, amelyeknél ez az ár a testmozgás. 

\section*{Zombies, Run!}
\label{zombiesrun}

A Zombies, Run! egy olyan népszerű futó alkalmazás, ahol a felhasználó egy zombi-apokalipszisbe csöppen bele túlélőként. 
Feladata, hogy minél több zsákmányt szerezzen futás közben, amivel egy bázist kell folyamatosan fejlesztenie, hogy az ott állomásozó túlélőket biztonságban tudhassa. 
A felhasználó kezdetben elindítja az alkalmazást, majd a futás közben elkezdődik a játék történetének mesélése. 
Ez nem folyamatos, a felhasználó tud közben zenét hallgatni. 

Futás közben találkozhatunk zombikkal, melyeket vagy a futás közben talált ellátmányért cserébe rázhatunk le, vagy egy rövid ideig 20\%-kal gyorsabban kell futni. 
Ez a folyamatos váltakozás a futás közben egyfajta intervallum edzésnek felel meg, melynek igen sok előnyös tulajdonsága van. 
Ugyanakkor az alkalmazás önmagában követi nyomon a felhasználó sportolási tevékenységét, a témámban elkészítendő játék ezzel szemben a már lesportolt tevékenységeket jutalmazná, melyet külső sport nyomkövető alkalmazásokból kapna meg. 
Más eltérés is mutatkozik a témámtól, mégpedig a játék stílusa. 
Míg a tervezett játékom szerepjáték jellegű, a Zombies, Run! gyűjtögetős játék, mely főleg történet központú. 

\section*{Tep}
\label{tep}

Meg kell továbbá említeni a Tep-et, amely magyar fejlesztésű. 
A Tep szintén motivációs sport-nyomkövető alkalmazás, mely a valós teljesítmények után ad jutalmat a játékban. 
A játék stílusa a népszerű Tamagotchi játékhoz hasonló, azaz egy virtuális állatkát kell gondoznunk mindennaposan. 
A kapott jutalmakat beválthatjuk a virtuális állatunk részére különböző étel, ital és dekoratív elemre is. 
Ez az alkalmazás is különálló sport-nyomkövetőként működik, azaz nem külső forrásból szerzi be az adatokat, ugyanakkor össze lehet kötni hordozható eszközökkel, a Fitbit-tel és a Jawbone eszközökkel. 
Annak ellenére, hogy a játék motivációs célt szolgál, a felhasználó kevésbé van ösztönözve a sportolásra, ugyanis ha nem sportol folyamatosan, az egyetlen változás, ami bekövetkezik, hogy ha az állatkát „simogatjuk”, akkor nem csóválja a farkát és éhezik az állat. 
Ezzel szemben az általam készített játék esetén amennyiben a felhasználó nem sportol, nem lesz képes fejlődni a játékban. 

\section*{Pokémon Go}
\label{pokemongo}

A Pokémon Go az RPG játékok egy speciális fajtába az MMORPG-be tartozik. 
A játék kizárólagosan csak mobil eszközre készült abból az okból kifolyólag, hogy közvetett vagy közvetlen módon sportolásra vagy legalább mozgásra ösztönözze az embereket. 
Emiatt szükség volt arra, hogy az eszköz, amin játsszanak, mobilis legyen. 
Mindezek mellett a játék félig a valóságban félig pedig a virtuális világban játszódik. 
A pontos pozíciónkat megjeleníti a térképen, ami a valós világ útjaira, épületeire alapszik. 
Annyiban viszont eltér, hogy a játék egyes elemeit például a pokémonokat (kitalált állatszerű lények) a virtuális világban a térképre helyezi, majd a felhasználónak a való világban fizikailag oda kell jutnia hozzá, hogy elkaphassa. 

\section{Az RPG játékokról általánosságban}
\label{rpgaltalanos}

A következő alfejezetben az RPG-t (role playing game), vagyis szerepjátékot fogom bemutatni. 
Ezek a fajta játékok arra épülnek, hogy a felhasználó egy karakter „szerepébe” bújik bele, őt irányítva végzi el a feladatokat, kalandozik a világban. 
Eredete az ókorba vezethető vissza, elődjének tekinthetőek a különböző harci játékok amelyek az ütközetek szimulálására szolgáltak. 
Az első szerepjáték az 1974-ben megjelent Dungeons \& Dragons volt, ami hasonló szabályrendszerrel rendelkezett, mint a napjainkban megjelenő RPG-k. 
Előnyük, hogy sok ember számára elérhetőek, ugyanis a játékhoz dobókockákra, papírra és képzelőerőre van szükség. 
A mesélőnek kinevezett személy vezeti végig a kalandozáson a többi szereplőt. 
Minden játékoshoz tartozik egy karakter, aki fölött rendelkezhet, illetve a karakterlapján vezetheti a statisztikákat és jellemzőket. 
Egy elvégzett feladat vagy küldetés után különbőző jutalmakat kaphatnak a karakterek, amelyek fejlődésük során egyre erősebbek lesznek, emiatt pedig sikerül elérniük a játék elején kitűzött céljukat. 
%karakterlaprol kép

Nem kellett sok idő ahhoz, hogy az RPG meghódítsa a számítógépes közeget is. 
Az 1970-es évek közepe után sorra jelentek meg a többfelhasználós kalandjátékok, amik a szerepjátékok szabályait követve nyújtottak szórakozási lehetőséget azoknak, akik rendelkeztek internetkapcsolattal. 
Ennek a mintájára napjainkban már nem csak webes felületen érhetőek el a hasonló típusú játékok, hanem az okostelefonok terjedésével, már mobil felületen is. 
A grafikai kártyák fejlődése következében az utóbbi két játékforma grafikai elemek felhasználásával szimulálja a különböző akciókat, amelyek a régi típusú szerepjátékban a képzeletre voltak bízva. 
Továbbá egy jelentős különbség még, hogy míg az eredeti szerepjátékokban a harcok kimenetele többnyire a szerencsén múlik, addig az online játékoknál különböző képletek és algoritmusok segítségével számolják ki, egy egy támadás mértékét. 
Feltehetőleg a komplexebb harcrendszer miatt alakult ki több változata a küzdelem lebonyolításának. 
Egy népszerű formája a „turn based” vagyis körökre osztott összecsapás. 
A játékos karaktere és az ellenfél felváltva támad, minden fél a saját körében dönt arról, hogy milyen cselekvést fog végrehajtani. 
Ez az akció lehet támadás, öngyógyítás vagy esetleg menekülési kísérlet. 
Ezen kívül vannak szimulált harcot implementáló játékok, ahol a harc kimenetele időközben nem befolyásolható.
Itt az összecsapás az ellenfél és a saját karakter tulajdonságpontjainak a felhasználásával kerül kiszámításra. 
A játékos a végeredményt látja csak, hogy sikerült e legyőzni az ellenséget vagy sem. 

Továbbá különbséget tehetünk abban, hogy az online felületen játszható játékok hosszú távon tudnak szórakozási lehetőséget biztosítani, nem szükséges a játékosok folyamatos jelenléte. 
Az online szerepjátékokban jellemzően nincs kitűzött végcél, a felhasználók kalandoznak, fejlődnek és egyre erősödő ellenfeleket győznek le. 
Ha a játékban nincsenek maximálisan elérhető értékek definiálva, akkor  csak a játékos kitartása és eltökéltsége szab határt a játék végének. 
A cél egy olyan játék megvalósítása volt, ami hosszú távon képes ösztönözni a felhasználót a sportolásra.

\newpage







