%2. fejezet

%2.1
\section{A videójátékok története}
\label{resz2_1}

% \Picture{A társasjáték}{2/1}{width=10cm}

Videójátékoknak nevezzük azokat a típusú játékokat, ahol a játékosok egy felhasználói felületen keresztüllépnek interakcióba a játékkal. Története az 1950-es évekre vezethető vissza, ebben az évtizedben kezdte el foglalkoztatni az embereket az az ötlet, hogy a szórakozás élményét elektromos eszközök segítségével érjék el. Az első videojáték egy asztalitenisz szimulátor volt, 1958-ban készítette el William Higinbotham és a „Tennis for Two” nevet viselte. A játék egy asztali analóg számítógépre készült el, és egy oszcilloszkópot használt a megjelenítésre.

\section{Sportjátékok}
\label{resz2_2}

Napjainkban nagy népszerűségnek örvendenek az olyan játékok, ahol a játékosok interakcióba tudnak lépni egymással, ez lehet együttműködés vagy egymás elleni versengés is. Mivel a versenyszellem velünk született tulajdonságunk, ezért a felhasználók minden tőlük telhetőt megtesznek azért, hogy a ranglista élére kerülhessenek. Erre alapozva sok játékkészítő nagyon kreatívan határozta meg a fejlődés árát. Az alábbiakban pár olyan játékot mutatok be, amelyeknél ez az ár a testmozgás.

\section*{Zombies, Run!}
\label{resz2_2_1}

A Zombies, Run! egy olyan népszerű futó alkalmazás, ahol a felhasználó egy zombi-apokalipszisbe csöppen bele túlélőként. Feladata, hogy minél több zsákmányt szerezzen futás közben, amivel egy bázist kell folyamatosan fejlesztenie, hogy az ott állomásozó túlélőket biztonságban tudhassa. A felhasználó kezdetben elindítja az alkalmazást, majd a futás közben elkezdődik a játék történetének mesélése. Ez nem folyamatos, a felhasználó tud közben zenéthallgatni

Futás közben találkozhatunk zombikkal, melyeket vagy a futás közben talált ellátmányért cserébe rázhatunk le, vagy egy rövid ideig 20%-kal gyorsabban kell futni. Ez a folyamatos váltakozás a futás közben megegyezik az intervallum edzésnek, melynek igen sok előnyös tulajdonsága van. Ugyanakkor az alkalmazás önmagában követi nyomon a felhasználó sportolási tevékenységét, a témámban elkészítendő játék ezzel szemben a már lesportolt tevékenységeket jutalmazná, melyet külső sport nyomkövető alkalmazásokból kapna meg. Más eltérés is mutatkozik a témámtól, mégpedig a játék stílusa. Míg a tervezett játékom szerepjáték jellegű, a Zombies, Run! gyűjtögetős játék, mely főleg történet központú.

\section*{Tep}
\label{resz2_2_2}

Meg kell továbbá említeni a Tep-et, amely magyar fejlesztésű. A Tep szintén motivációs sport-nyomkövető alkalmazás, mely a valós teljesítmények után ad jutalmat a játékban. A játék stílusa a népszerű Tamagotchi játékhoz hasonló, azaz egy virtuális állatkát kell gondoznunk mindennaposan. A kapott jutalmakat beválthatjuk a virtuális állatunk részére különböző étel, ital és dekoratív elemre is. Ez az alkalmazás is különálló sport-nyomkövetőként működik, azaz nem külső forrásból szerzi be az adatokat, ugyanakkor össze lehet kötni hordozható eszközökkel, a Fitbit-tel és a Jawbone eszközökkel. Annak ellenére, hogy a játék motivációs célt szolgál, a felhasználó kevésbé van ösztönözve a sportolásra, ugyanis ha nem sportol folyamatosan, az egyetlen változás, ami bekövetkezik, hogy ha az állatkát „simogatjuk”, akkor nem csóválja a farkát és éhezik az állat. Ezzel szemben az általam készített játék esetén amennyiben a felhasználó nem sportol, nem lesz képes fejlődni a játékban.

\section*{Pokémon Go}
\label{resz2_2_3}

A Pokémon Go az RPG játékok egy speciális fajtába az MMORPG-be tartozik. A játék kizárólagosan csak mobil eszközre készült abból az okból kifolyólag, hogy közvetett vagy közvetlen módon sportolásra vagy legalább mozgásra ösztönözze az embereket. Emiatt szükség volt arra, hogy az eszköz, amin játsszanak, mobilis legyen. Mindezek mellett a játék félig a valóságban félig pedig a virtuális világban játszódik. A pontos pozíciónkat megjeleníti a térképen, ami a valós világ útjaira, épületeire alapszik. Annyiban viszont eltér, hogy a játék egyes elemeit például a pokémonokat (kitalált állatszerű lények) a virtuális világban a térképre helyezi, majd a felhasználónak a való világban fizikailag oda kell jutnia hozzá, hogy elkaphassa.

\section{Az RPG játékokról általánosságban}
\label{resz2_3}

Ezek a fajta játékok arra épülnek, hogy a felhasználó egy karakter „szerepébe” bújik bele, őt irányítva végzi el a feladatokat, kalandozik a világban. A játék egyik fő tényezője a játékosok szintje. Ezt játékoktól függően tudjuk különböző tevékenységekért úgynevezett tapasztalat pontokkal növelni. Ezek a tevékenységek főleg küldetésekben jelennek meg a játékban. A küldetéshez tartozik egy leírás az elvégzendő feladatról, esetleg annak jutalmáról. A játékhoz tartozik még a harcrendszer. Ez is több féle módon valósulhat meg, a „turn based” vagyis körökre osztott harcrendszer a leggyakoribb. A karakterünk és az ellenfél felváltva támad, a saját körünkben döntünk arról, hogy milyen akciót szeretnénk aktiválni. Ez lehet támadás, gyógyítás stb. Ezen kívül vannak szimulált harcot implementáló játékok. Itt a harc az ellenfél és a saját karakter tulajdonságpontjainak a felhasználásával kerül kiszámításra. A játékos a végeredményt látja csak, hogy sikerült e legyőzni az ellenséget vagy sem.

[Ez a rész még bővíthető]
\newpage







