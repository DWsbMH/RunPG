Az elkészült játék teljesíti a felé támasztott követelményeket ugyan, de ahogyan minden egyéb más program, ez sem tökéletes. 
Előfordulhatnak olyan nem várt hibák, amelyek a fejlesztés és tesztelés során nem jelentkeztek. 
Ezek a hibák lehetnek a játék nem megfelelő használata miatt, a külső függőségekben található hibák miatt, melyeket az adott könyvtárak nem kezeltek le, vagy akár az operációs rendszer hibájából. 
Az ilyen hibák számát a továbbfejlesztés során ha nem is lehet nullára csökkenteni, de a lehetséges legkisebb értékre kell minimalizálni. 
A feladatspecifikus programokkal ellentétben egy játékhoz megannyi továbbfejlesztési ötlet juthat a fejlesztő de akár a játékos eszébe is. 
Nem történt ez másként most sem, már a fejlesztési szakaszban kínálkoztak olyan továbbfejlesztési lehetőségek, amiket érdemes lenne megvalósítani a szórakoztatóbb játékélmény érdekében. 

\subsection*{További sport-trackerek integrációja}
\label{sporttrackerintegration}
A játék szempontjából nagyon fontos tényező, hogy a játékos elérje az általa használt sport-tracker alkalmazással rögzített adatokat. 
Jelenleg integrálásra került a két legnagyobb ilyen tracker, de természetesen vannak olyan emberek akik nem ezeket használják. 
Ahhoz, hogy ezeket az embereket is elérhesse a játék, a jövőben minél több sport-trackert kellene integrálni a játékba. 

\subsection*{Online játékszolgáltatások}
\label{onlinegameingservice}
Első ilyen ötletként az online játékszolgáltatások integrációja merült fel, a platformból adódóan főként a Google Play Games szolgáltatásé, melyhez külön SDK is tartozik. 
Amennyiben integrálásra kerül, lehetőség nyílik valós idejű többjátékos módra, az adatok felhőbe való mentésére, közösségi és nyilvános ranglisták készítésére, kitüntetések osztására és illegális szoftverhasználat elleni védekezésre is. 
A többjátékos mód esetén a játékban nem csak az adott játékos karaktere jelenne meg, hanem minden online, az adott helyen tartózkodó játékosé is, így egymással kommunikálni is tudnának a játékosok, mely a legtöbb széles körben elterjedt online szerepjáték alapvető eleme. 
Az adatok felhőbe mentésével elérhetővé válik hogy a felhasználó egy eszközön elért eredményeit egy másik eszközön folytathassa, így készülékcsere esetén nem kellene elölről kezdeni a játékot. 
A ranglisták és kitüntetések bevezetésével a játékosok közt kialakulna egy természetes versenyszellem, hiszen mindenki szeretne minél jobb helyezést elérni, főleg, ha azt ismerőseinek is meg tudja mutatni. 
Az illegális szoftverhasználat segítségével meggátolhatóvá válnak a játékkal való visszaélések, így a tisztességesen játszó emberek munkája nem veszik kárba.  

\subsection*{Játékmechanika javítása, új elemek hozzáadása}
\label{jatekmechanika}
A játékélmény fokozására a meglévőeken kívül új játékmechanikai elemeket lenne érdemes bevezetni, és a meglévőeket folyamatosan frissíteni, finomhangolni. 
Ilyen új elemek lehetnek például zárt ajtók, amelyeken a játékos csak az ajtóhoz tartozó kulccsal tudna átjutni, a játékos számára az ugrás képességének implementálása, hogy a magasabban található elemekhez is hozzáférhessen. 
Továbbá a meglévők mellé új fegyverek hozzáadása, illetve más kiegészítők elkészítését, melyek mindegyike a harc kimenetelét befolyásolnák. 
A játékos figyelmének napi szinten való fenntartásához létre lehetne hozni egy új épületet, ahol mindennapos gyakorisággal küldetéseket vállalhatna. 
Ezen küldetések teljesítési feltételét és jutalmát megjelenítenénk a felhasználó számára, így a játékos eldöntheti, hogy képes-e teljesíteni a feladatot, vagy megéri a jutalomért elvégeznie azt. 
A játékosok számára saját házat lehetne készíteni, mely többféle feladatot is betölthetne, mint például a különböző tárgyak tárolása vagy éppen olyan különleges elemek elhelyezése benne, amiknek permanens pozitív hatása lenne a játékosra nézve. 

\subsection*{Játékosok közötti küzdelem}
\label{pvp}
További ötletként felmerült a játékosok közötti küzdelem lebonyolítása, mely során kiderülne mely felhasználó karaktere az erősebb. 
Ez a plusz funkció nagyban növelheti a játékosok sportolásra való motivációját, hisz a vesztes fél fel akarná erősíteni a karakterét, hogy a következő összecsapásban ő kerekedjen felül, a nyertes játékos pedig azért sportolna még többet, hogy a karakterét még tovább növelve megvédhesse a nyertes pozícióját. 
Természetesen különböző korlátokat kellene behozni, elkerülve hogy egy nagy szintű karakter ne küzdhessen meg csak bizonyos szinthatáron belüli karakterrel. 
Továbbá korlátozni lehetne az ilyen típusú harcok gyakoriságát is, vagy akár feltételhez kötni, esetleg valamilyen minimális szintű testmozgás esetén válna elérhető. 

\subsection*{Pályaszerkesztő}
\label{palyaszerkeszto}
Érdemes lenne megvalósítani egy online pályaszerkesztő szolgáltatást is. 
Ennek segítségével bárki készíthetne a játékon belül játszható pályákat, ezeknek szerkesztés közben meg lehetne adni minden fontosabb paramétert, mint amilyen a méretei, nehézsége és a teljesítésének feltételét. 
A szerkesztés közben el lehetne helyezni adott pozícióra ellenfeleket és díszítő elemeket. 
Az így elkészült pályák igen szórakoztató kihívásnak tűnnek, hiszen a szerkesztők kreativitásának köszönhetően olyan megoldások is születhetnének, melyek addig nem jelentek meg a játékban. 

\subsection*{Szerver}
\label{szerver}
Az utóbbi két továbbfejlesztési lehetőséghez el kellene készíteni egy online kiszolgálót, így nem játékos eszközét terhelve ezen feladatok végrehajtásával, valamint ezen keresztül biztosítva folyamatos működésüket. 
Ezen kívül a szerveren ki lehetne alakítani egy fórumot is, ahol a játékosok megoszthatnák tapasztalataikat a többi játékossal, segítve a kezdőket. 








