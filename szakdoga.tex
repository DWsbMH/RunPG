%% Based on a TeXnicCenter-Template by Tino Weinkauf.
%%%%%%%%%%%%%%%%%%%%%%%%%%%%%%%%%%%%%%%%%%%%%%%%%%%%%%%%%%%%%


%%%%%%%%%%%%%%%%%%%%%%%%%%%%%%%%%%%%%%%%%%%%%%%%%%%%%%%%%%%%%
%% HEADER
%%%%%%%%%%%%%%%%%%%%%%%%%%%%%%%%%%%%%%%%%%%%%%%%%%%%%%%%%%%%%
\documentclass[a4paper,oneside,10pt]{report}
% Alternative Options:
%	Paper Size: a4paper / a5paper / b5paper / letterpaper / legalpaper / executivepaper
% Duplex: oneside / twoside
% Base Font Size: 10pt / 11pt / 12pt

\usepackage{hegyhati}


%%%%%%%%%%%%%%%%%%%%%%%%%%
\usepackage{t1enc} 
\usepackage{lmodern} 




%%%%%%%%%%%%%%%%%%%%%%%%%%%%%%%%%%%%%%%%%%%%%%%%%%%%%%%%%%%%%
%% DOCUMENT
%%%%%%%%%%%%%%%%%%%%%%%%%%%%%%%%%%%%%%%%%%%%%%%%%%%%%%%%%%%%%
\begin{document}



\begin{titlepage}
\begin{center}
\Large
Pannon Egyetem

\vspace{10mm}
Műszaki Informatikai Kar

\vspace{10mm}
Rendszer- és Számítástudományi Tanszék

\vspace{10mm}
Mérnökinformatikus MSc

\vspace{40mm}
\huge
DIPLOMAMUNKA

\vspace{10mm}
\LARGE
"Free to play, run to win" játék Androidra

\vspace{10mm}
\Large
Nyitrai Tamás

\vspace{40mm}
Témavezető: Dr. Hegyháti Máté

\vspace{10mm}
2016
\normalsize
\end{center}
\end{titlepage}



\pagestyle{empty} %No headings for the first pages.



\newpage
\Large
\begin{center}
	\textbf{KÖSZÖNETNYILVÁNÍTÁS}
\end{center}
\normalsize
\noindent
Meg szeretném köszönni édesanyámnak támogatását és folyamatos bátorítását. Továbbiakban köszönettel tartozom témavezetőmnek, Dr. Hegyháti Máténak, aki ötleteivel és tanácsaival végig a helyes úton tartott.

Továbbá szeretném megköszönni Hollósi Tamásnak a dream-iso-android elkészítőjének a segítségét, akihez bátran fordulhattam, ha bármi kérdésem volt a játékmotorral kapcsolatban. Valamint köszönöm Böröndi Evelinnek a grafikák elkészítésében vállalt segítségét.
\newpage
\Large
\begin{center}
	\textbf{TARTALMI ÖSSZEFOGLALÓ}
\end{center}
\normalsize
\noindent


Tartalmi összefoglaló...

\textbf{Kulcsszavak:} dream-iso-droid, Android, testmozgás, szerepjáték

\newpage

\Large
\begin{center}
	\textbf{ABSTRACT}
\end{center}
\normalsize
\noindent
Angol tartalmi összefoglaló...

\textbf{Keywords:} dream-iso-droid, Android, sports, RPG game
\tableofcontents
\newpage
\listoffigures
\newpage

%======================================================================


%% Title Page %%%%%%%%%%%%%%%%%%%%%%%%%%%%%%%%%%%%%%%%%%%%%%%
%% ==> Write your text here or include other files.

%% The simple version:
%\title{Címoldal}
%\author{Nyitrai Tamás}
%\date{} %%If commented, the current date is used.
%\maketitle

%% Inhaltsverzeichnis %%%%%%%%%%%%%%%%%%%%%%%%%%%%%%%%%%%%%%%
%\tableofcontents %Table of contents
%\cleardoublepage %The first chapter should start on an odd page.

\pagestyle{plain} %Now display headings: headings / fancy / ...



%% Chapters %%%%%%%%%%%%%%%%%%%%%%%%%%%%%%%%%%%%%%%%%%%%%%%%%
%% ==> Write your text here or include other files.

%\input{intro} %You need a file 'intro.tex' for this.


%%%%%%%%%%%%%%%%%%%%%%%%%%%%%%%%%%%%%%%%%%%%%%%%%%%%%%%%%%%%%

%1. fejezet
\chapter{Bevezetés}
\label{bev}
%1. fejezet
Az okos-telefonok térhódítása miatt már hazánkban is a lakosság fele rendelkezik valamilyen okos eszközzel, ez a szám pedig a jövőben egyre csak növekedni fog. 
A mindennapi élet megkönnyítésére rengeteg féle alkalmazás születik napról napra. 
Egyetlen eszközön olvashatunk újságot, tudakozódhatunk a közlekedésről, vehetünk ebédet magunknak, vagy unaloműzésként játszhatunk. 
Manapság minden korosztály talál kedvére való játékot, legyen szó akár ingyenes akár fizetős verzióról. 
Napjainkban igencsak elterjedtek az olyan játékok ahol a felhasználók interakciókba léphetnek egymással. 
Ezek túlnyomó többsége az úgynevezett micro-paymentekre alapszik, ahol a játékosok csekély összegekért cserébe előnyökhöz, könnyítésekhez juthatnak. 
A „free to play, pay to win” kifejezést azokra a játékokra szokták használni, ahol az előbb említett vásárlások nélkül képtelenség megnyerni a játékot, mert túlzottan befolyásolják a játékosok fejlődését.

Egyre több ember életéből hiányzik napjainkban a rendszeres testmozgás. 
Sokan választják a séta és a bicikli helyett az autós vagy a tömegközlekedést, főleg kényelmi szempontokból. 
A technológia fejlődésével pedig egyre több olyan eszköz jön létre, amik az emberek életének kényelmesebbé tételét szolgálja. 
Ha rendelkezünk okostelefonnal, ma már a nagybevásárlást is el tudjuk intézni pár kattintással otthonról. 
Az ilyen alkalmazások célja az volt, hogy kényelmesebbé tegyék mindennapjainkat, nem az, hogy elkényelmesítsenek minket. 
Az, hogy egyre több időt töltünk ezen eszközök előtt, csak súlyosbítja a rendszeres testmozgás hiányát. 
Sajnálatos módon a felhasználók nagy részét nem motiválja önmagában eléggé az, hogy a mozgás jót tesz az egészségének, szükség lehet valamilyen fajta ösztönző módszerre. 
Ezek többféle módon is megnyilvánulhatnak, léteznek különböző virtuális díjazások, sportolásért járó pontok, amik később tárgynyereményekre válthatóak, sőt vannak tényleges pénzzel való díjazások is.

Célom a két trend összekapcsolása oly módon, hogy a játékban a gyorsabb fejlődést nem pénzkifizetéssel, hanem sportolással lehessen kiváltani. 

A 2. fejezetben ismertetem az elkészült játék hátterét, bemutatok különböző játéktípusokat. 

A 3. fejezetben ismertetem a program tervezett funkcionalitását és követelményeit. 
Illetve kitérek a játék egy fontos alapelemére az általam választott dream-iso-droid játékmotorra. 

A 4. fejezetben bemutatom az elkészült alkalmazást felhasználói szemmel. 

Az 5. fejezetben részletesen kifejtem a megvalósított játék fő alkotóelemeit, és azok implementációját. 

Végül a 6. fejezetben bemutatok pár lehetséges továbbfejlesztési ötletet, melyeket érdemes lenne megvalósítani. 

%2. fejezet
\chapter{Hasonló fejlesztések, szerepjátékokról általánosan}
\label{bem}
%2. fejezet

%2.1
\section{A videójátékok története}
\label{resz2_1}

% \Picture{A társasjáték}{2/1}{width=10cm}

Videójátékoknak nevezzük azokat a típusú játékokat, ahol a játékosok egy felhasználói felületen keresztüllépnek interakcióba a játékkal. Története az 1950-es évekre vezethető vissza, ebben az évtizedben kezdte el foglalkoztatni az embereket az az ötlet, hogy a szórakozás élményét elektromos eszközök segítségével érjék el. Az első videojáték egy asztalitenisz szimulátor volt, 1958-ban készítette el William Higinbotham és a „Tennis for Two” nevet viselte. A játék egy asztali analóg számítógépre készült el, és egy oszcilloszkópot használt a megjelenítésre.

\section{Sportjátékok}
\label{resz2_2}

Napjainkban nagy népszerűségnek örvendenek az olyan játékok, ahol a játékosok interakcióba tudnak lépni egymással, ez lehet együttműködés vagy egymás elleni versengés is. Mivel a versenyszellem velünk született tulajdonságunk, ezért a felhasználók minden tőlük telhetőt megtesznek azért, hogy a ranglista élére kerülhessenek. Erre alapozva sok játékkészítő nagyon kreatívan határozta meg a fejlődés árát. Az alábbiakban pár olyan játékot mutatok be, amelyeknél ez az ár a testmozgás.

\section*{Zombies, Run!}
\label{resz2_2_1}

A Zombies, Run! egy olyan népszerű futó alkalmazás, ahol a felhasználó egy zombi-apokalipszisbe csöppen bele túlélőként. Feladata, hogy minél több zsákmányt szerezzen futás közben, amivel egy bázist kell folyamatosan fejlesztenie, hogy az ott állomásozó túlélőket biztonságban tudhassa. A felhasználó kezdetben elindítja az alkalmazást, majd a futás közben elkezdődik a játék történetének mesélése. Ez nem folyamatos, a felhasználó tud közben zenéthallgatni

Futás közben találkozhatunk zombikkal, melyeket vagy a futás közben talált ellátmányért cserébe rázhatunk le, vagy egy rövid ideig 20%-kal gyorsabban kell futni. Ez a folyamatos váltakozás a futás közben megegyezik az intervallum edzésnek, melynek igen sok előnyös tulajdonsága van. Ugyanakkor az alkalmazás önmagában követi nyomon a felhasználó sportolási tevékenységét, a témámban elkészítendő játék ezzel szemben a már lesportolt tevékenységeket jutalmazná, melyet külső sport nyomkövető alkalmazásokból kapna meg. Más eltérés is mutatkozik a témámtól, mégpedig a játék stílusa. Míg a tervezett játékom szerepjáték jellegű, a Zombies, Run! gyűjtögetős játék, mely főleg történet központú.

\section*{Tep}
\label{resz2_2_2}

Meg kell továbbá említeni a Tep-et, amely magyar fejlesztésű. A Tep szintén motivációs sport-nyomkövető alkalmazás, mely a valós teljesítmények után ad jutalmat a játékban. A játék stílusa a népszerű Tamagotchi játékhoz hasonló, azaz egy virtuális állatkát kell gondoznunk mindennaposan. A kapott jutalmakat beválthatjuk a virtuális állatunk részére különböző étel, ital és dekoratív elemre is. Ez az alkalmazás is különálló sport-nyomkövetőként működik, azaz nem külső forrásból szerzi be az adatokat, ugyanakkor össze lehet kötni hordozható eszközökkel, a Fitbit-tel és a Jawbone eszközökkel. Annak ellenére, hogy a játék motivációs célt szolgál, a felhasználó kevésbé van ösztönözve a sportolásra, ugyanis ha nem sportol folyamatosan, az egyetlen változás, ami bekövetkezik, hogy ha az állatkát „simogatjuk”, akkor nem csóválja a farkát és éhezik az állat. Ezzel szemben az általam készített játék esetén amennyiben a felhasználó nem sportol, nem lesz képes fejlődni a játékban.

\section*{Pokémon Go}
\label{resz2_2_3}

A Pokémon Go az RPG játékok egy speciális fajtába az MMORPG-be tartozik. A játék kizárólagosan csak mobil eszközre készült abból az okból kifolyólag, hogy közvetett vagy közvetlen módon sportolásra vagy legalább mozgásra ösztönözze az embereket. Emiatt szükség volt arra, hogy az eszköz, amin játsszanak, mobilis legyen. Mindezek mellett a játék félig a valóságban félig pedig a virtuális világban játszódik. A pontos pozíciónkat megjeleníti a térképen, ami a valós világ útjaira, épületeire alapszik. Annyiban viszont eltér, hogy a játék egyes elemeit például a pokémonokat (kitalált állatszerű lények) a virtuális világban a térképre helyezi, majd a felhasználónak a való világban fizikailag oda kell jutnia hozzá, hogy elkaphassa.

\section{Az RPG játékokról általánosságban}
\label{resz2_3}

Ezek a fajta játékok arra épülnek, hogy a felhasználó egy karakter „szerepébe” bújik bele, őt irányítva végzi el a feladatokat, kalandozik a világban. A játék egyik fő tényezője a játékosok szintje. Ezt játékoktól függően tudjuk különböző tevékenységekért úgynevezett tapasztalat pontokkal növelni. Ezek a tevékenységek főleg küldetésekben jelennek meg a játékban. A küldetéshez tartozik egy leírás az elvégzendő feladatról, esetleg annak jutalmáról. A játékhoz tartozik még a harcrendszer. Ez is több féle módon valósulhat meg, a „turn based” vagyis körökre osztott harcrendszer a leggyakoribb. A karakterünk és az ellenfél felváltva támad, a saját körünkben döntünk arról, hogy milyen akciót szeretnénk aktiválni. Ez lehet támadás, gyógyítás stb. Ezen kívül vannak szimulált harcot implementáló játékok. Itt a harc az ellenfél és a saját karakter tulajdonságpontjainak a felhasználásával kerül kiszámításra. A játékos a végeredményt látja csak, hogy sikerült e legyőzni az ellenséget vagy sem.

[Ez a rész még bővíthető]
\newpage









%3. fejezet
\chapter{Követelmények, technológiák}
\label{fejlesztes}
%3. fejezet

A fejezetben szó esik a szoftverrel szemben támasztott követelményekről, valamint a felhasznált programok technológiák kerülnek bemutatásra. 

%3.1
\section{Követelmények}
\label{resz3_1}

Az alkalmazással szemben különféle követelményeket támasztok, melyeknek mindenképp meg kell felelnie, melyek a következőek:

\begin{itemize}
	\item Motiválás:
	\\
	Első és legfontosabb követelmény, hogy képes legyen az embereket ösztönöznie a sportolásra. Ezt minél érdekesebb és izgalmasabb játékmechanikai elemekkel kívánom elérni.
	\item Kiterjeszthetőség:
	\\
	Az elkészítendő játék nem különálló tracker alkalmazásként fog működni, önmagában nem lesz képes mérni a sporttevékenységeket. Minden esetben más tracker szolgáltatások által elmentett tevékenységet fog lekérni az alkalmazás. Emiatt fontos, hogy minél több alkalmazástól tudjon lementett sporttevékenységeket lekérni. Amiatt is fontos lenne minél több tracker alkalmazás támogatása, mivel azon felhasználók, akik már régebb óta sportolnak, ne kelljen a számukra bevált sport nyomkövető szolgáltatást lecserélni.
	\item Érdeklődés fenntartás
	\\
	Miután a játék felkeltette a felhasználó figyelmét, el kell érni, hogy tovább játsszon vele. Ahogy a korábbiakban említettem, a játékmechanikai elemekkel és különböző kihívásokkal szeretném megvalósítani, melyeket lejjebb fejtek ki bővebben.

	\item Gyors, sok eszközön, ezt majd ki kell fejteni....
	\\
	Kifejtés..
\end{itemize}


%3.1.1
\section*{Funkcionális követelmények}
\label{resz3_1_1}

Az alábbiakban a főbb funkcionális követelmények kerülnek bemutatásra.

\begin{itemize}
	\item A játéknak képesnek kell lennie csatlakozni sport-nyomkövető alkalmazásokhoz. Lehetőleg minél több alkalmazást kell támogatnia a játéknak, mivel így nagyobb lehet az elérhető potenciális játékos-közösség is.
	\item A minél nagyobb számú támogatottság elérése érdekében az alkalmazásnak mindenképp támogatnia kell az OAuth szabványt. A legtöbb, ha nem minden sport tracker alkalmazás ezt a szabványt használja a külső alkalmazásokkal való kapcsolódásra. Amikor csatlakozni szeretnénk az adott profilunkhoz külső alkalmazásból, meg kell adni az engedélyt, hogy az alkalmazás mely adatainkhoz férjen hozzá.
	\item Csatlakozás után az alkalmazásnak le kell töltenie a felhasználó legújabb sport tevékenységeit. Erőforrás takarékosság szempontjából először meg kell bizonyosodni, hogy van-e új tevékenység. Törekedni kell, hogy a felhasználóhoz tartozó összes adatot csak az első csatlakozás alkalmával, vagy más eszközön való bejelentkezés esetén töltsük le.
	\item Ha a felhasználó nem csatlakozik más sport tracker alkalmazáshoz, akkor bejelentkezés esetén kell megbizonyosodni, hogy van-e csatlakoztatott sport trackerek esetén történt-e új tevékenység felvitel. Amennyiben igen, úgy csak ezeket az új tevékenységeket kell letölteni.
	\item A letöltött adatokat az Androidos eszközökön kell tárolni. Erre azért van szükség, hogy a későbbiekben a régebbi eseményeket, vagy a már jutalmazott tevékenységekért ne adjunk újra bónuszt. Ennél a pontnál szembe kell nézni a ténnyel, hogy a különböző sport trackerek bizonyos adatokat másként tárolnak, vagy teljesen hiányoznak. Emiatt létre kell hozni egy olyan általános adatbázis táblát, amelyben minden olyan adatot tárolunk, amelyekért jutalmat akarunk osztani a játékos számára. A későbbiekben támogatottságot nyerő sport trackereknek így valamilyen módon szolgáltatnia kell legalább azokat az információkat, amik ebben az adattáblában kapnak helyet. Az adattípusuk különbözhet, és amennyiben egy bizonyos adatot nem szolgáltat, de más adatokból származtatni lehet, úgy az nem okozhat akadályt.
	\item Bejelentkezés után az újonnan letöltött adatok alapján a felhasználó staminát (kitartást) kap. Egy játékosnak maximum 100 staminája lehet. A kapott stamina mennyisége összhangban kell lennie ezzel a maximális értékkel, a játékos szintjével, és a tevékenységben szereplő adatok nagyságával. Azaz az alacsony és magas szintű felhasználóknak is egyaránt élvezetesnek kell maradnia a játéknak, nem szabad se túl sokat, se túl keveset kapni. Túl sok stamina esetén nagyon könnyen haladhatna a felhasználó a játékban, így egy idő után beleunna, túl kevés esetén viszont a folyamatosan túl nagy kihívást jelentő és csak nagy megerőltetést jelentő tevékenységek szintén ugyanezt a hatást érnék el.
	\item A jutalomként megkapott staminát a felhasználó a játékosa fejlődésére használhatja fel különböző módokon. Az egyik ilyen mód a világban való "barangolás", ami közben szörnyek támadhatnak a játékosra, amelyeket legyőzve játékbeli pénzt és tapasztalati pontot kap a játékos. A másik mód küldetések vállalása, amelyet a felhasználónak kell ténylegesen sportolva teljesíteni, és csak a teljesítése után kapja meg az érte járó játékbeli jutalmat.
	\item A játékos ezen kívül rendelkeznie kell tulajdonságokkal is, melyek a szörnyek elleni csatában segíthetnek számára. Tulajdonságot növeli szintlépéssel vagy valamilyen kirívó sportteljesítményért cserébe lenne érdemes megengedni.
	\item További tárgyakat is érdemes lenne megvalósítani a játékos számára, melyek védelemmel vagy támadóerővel növelhetnék a játékos erejét.
\end{itemize}


\section{Felhasznált technológiák}
\label{resz3_2}

\section*{Játékmotorok}
\label{resz3_2_1}

Játékmotornak nevezzük a játékok - legyen az akár számítógépre vagy konzolra készült – azon részét, amely a program alapjául szolgáló technológiát adja. Szerepe, hogy megkönnyítse a fejlesztést illetve segítségével több platformon is futtatható lesz a játék.

A fejlesztés megkezdése előtt több fajta játékmotort is megvizsgáltam abból a célból, hogy kiválasszam a legmegfelelőbbet a diplomamunkám elkészítéséhez. 

A fő szempontom az volt, hogy ingyenesen elérhető legyen, illetve illeszkedjen a választott játéktípus játékmenetéhez.

A két legnépszerűbb motorral kezdtem az ismerkedést, a Unity és az Unreal engine-ekkel. Mivel a programomat Android platformon terveztem elkészíteni, amit Java nyelven kell implementálni, ezek a motorok pedig a C++ nyelvet támogatják, így nem lehet közvetlenül Java nyelven használni őket ezért hamar kiestek. Méretük alapján túl nagynak is bizonyultak volna egy ilyen kisebb méretű projekthez. A következő játékmotor, amit megvizsgáltam a HexEngine volt, amit Szabó László készített el MSc diplomamunkájaként. Ez a motor kifejezetten körökre osztott játékokra lett kifejlesztve, amivel az általam írt játék is rendelkezik, viszont a játéktér hatszögű blokkokra van osztva, amivel megbonyolította volna a közlekedést a játékon belül.

A választásom így Hollósi Tamás által készített dream-iso-droid játékmotorra esett, amit témavezetőm ismertetett meg velem. Mivel készítője elérhető közelségben volt, ezért könnyebben sikerült megismerkednem a motor nyújtotta funkciókkal.

\Picture{Játékmotorok összehasonlítása}{3/2}{width=10cm}

A dream-iso-droid egy olyan speciális játékmotor, ami kifejezetten Android platformra készült és a két dimenziós izometrikus nézetet támogatja. Ez a két funkciója pontosan megfelelt az elvárásaimnak, amit a játékmotor felé támasztottam, aminek segítségével fejleszteni szerettem volna az RPG játékomat.







%% <== End of hints
%%%%%%%%%%%%%%%%%%%%%%%%%%%%%%%%%%%%%%%%%%%%%%%%%%%%%%%%%%%%%


%%%%%%%%%%%%%%%%%%%%%%%%%%%%%%%%%%%%%%%%%%%%%%%%%%%%%%%%%%%%%
%% BIBLIOGRAPHY AND OTHER LISTS
%%%%%%%%%%%%%%%%%%%%%%%%%%%%%%%%%%%%%%%%%%%%%%%%%%%%%%%%%%%%%
%% A small distance to the other stuff in the table of contents (toc)
\addtocontents{toc}{\protect\vspace*{\baselineskip}}


\bibliographystyle{mybibstyle}
\bibliography{cite}

%% The List of Figures
%\clearpage
%\addcontentsline{toc}{chapter}{List of Figures}
%\listoffigures

%% The List of Tables
%\clearpage
%\addcontentsline{toc}{chapter}{List of Tables}
%\listoftables


\newpage

\Large
\begin{center}
	\textbf{MELLÉKLET}
\end{center}
\normalsize
\noindent
A mellékelt CD könyvtárszerkezete


% \begin{itemize}
%    \item \textbf{Dokumentum}
%    \begin{itemize}
%        \item \textbf{Forrás} - A szakdolgozat szerkeszthető formátumban
%        \item \textbf{Hivatkozások} - A szakdolgozatban lévő internetes hivatkozások letöltve
%        \item szakdolgozat.pdf
%    \end{itemize}
%    \item \textbf{Forrás} - A program forrásállománya
%    \begin{itemize}
%        \item \textbf{HexEngine}
%        \item \textbf{HexEngine-android}
%        \item \textbf{HexEngine-desktop}
%    \end{itemize}
%    \item \textbf{Program} - A futtatható program
%    \begin{itemize}
%        \item \textbf{bin} - A program grafikai és konfigurációs fájljai ami szükségesek az indításhoz
%        \item \textbf{hav} - A program a mentéseket tárolja itt
%        \item \textbf{java\_telepito} - A Java környezet telepítői
%        \item \textbf{libs} - Futtatáshoz kapcsolódó java fájlok
%        \item starter.jar
%    \end{itemize}
% \end{itemize}





%%%%%%%%%%%%%%%%%%%%%%%%%%%%%%%%%%%%%%%%%%%%%%%%%%%%%%%%%%%%%
%% APPENDICES
%%%%%%%%%%%%%%%%%%%%%%%%%%%%%%%%%%%%%%%%%%%%%%%%%%%%%%%%%%%%%
\appendix
%% ==> Write your text here or include other files.

%\input{FileName} %You need a file 'FileName.tex' for this.


\end{document}

