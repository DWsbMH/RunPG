%% Based on a TeXnicCenter-Template by Tino Weinkauf.
%%%%%%%%%%%%%%%%%%%%%%%%%%%%%%%%%%%%%%%%%%%%%%%%%%%%%%%%%%%%%


%%%%%%%%%%%%%%%%%%%%%%%%%%%%%%%%%%%%%%%%%%%%%%%%%%%%%%%%%%%%%
%% HEADER
%%%%%%%%%%%%%%%%%%%%%%%%%%%%%%%%%%%%%%%%%%%%%%%%%%%%%%%%%%%%%
\documentclass[a4paper,oneside,10pt]{report}
% Alternative Options:
%	Paper Size: a4paper / a5paper / b5paper / letterpaper / legalpaper / executivepaper
% Duplex: oneside / twoside
% Base Font Size: 10pt / 11pt / 12pt

\usepackage{hegyhati}


%%%%%%%%%%%%%%%%%%%%%%%%%%
\usepackage{t1enc} 
\usepackage{lmodern} 




%%%%%%%%%%%%%%%%%%%%%%%%%%%%%%%%%%%%%%%%%%%%%%%%%%%%%%%%%%%%%
%% DOCUMENT
%%%%%%%%%%%%%%%%%%%%%%%%%%%%%%%%%%%%%%%%%%%%%%%%%%%%%%%%%%%%%
\begin{document}



\begin{titlepage}
\begin{center}
\Large
Pannon Egyetem

\vspace{10mm}
Műszaki Informatikai Kar

\vspace{10mm}
Rendszer- és Számítástudományi Tanszék

\vspace{10mm}
Mérnökinformatikus MSc

\vspace{40mm}
\huge
DIPLOMAMUNKA

\vspace{10mm}
\LARGE
"Free to play, run to win" játék Androidra

\vspace{10mm}
\Large
Nyitrai Tamás

\vspace{40mm}
Témavezető: Dr. Hegyháti Máté

\vspace{10mm}
2016
\normalsize
\end{center}
\end{titlepage}



\pagestyle{empty} %No headings for the first pages.



\newpage
\Large
\begin{center}
	\textbf{KÖSZÖNETNYILVÁNÍTÁS}
\end{center}
\normalsize
\noindent
Meg szeretném köszönni édesanyámnak támogatását és folyamatos bátorítását. Továbbiakban köszönettel tartozom témavezetőmnek, Dr. Hegyháti Máténak, aki ötleteivel és tanácsaival végig a helyes úton tartott.

Továbbá szeretném megköszönni Hollósi Tamásnak a dream-iso-android elkészítőjének a segítségét, akihez bátran fordulhattam, ha bármi kérdésem volt a játékmotorral kapcsolatban. Valamint köszönöm Böröndi Evelinnek a grafikák elkészítésében vállalt segítségét.
\newpage
\Large
\begin{center}
	\textbf{TARTALMI ÖSSZEFOGLALÓ}
\end{center}
\normalsize
\noindent


Tartalmi összefoglaló...

\textbf{Kulcsszavak:} dream-iso-droid, Android, testmozgás, szerepjáték

\newpage

\Large
\begin{center}
	\textbf{ABSTRACT}
\end{center}
\normalsize
\noindent
Angol tartalmi összefoglaló...

\textbf{Keywords:} dream-iso-droid, Android, sports, RPG game
\tableofcontents
\newpage
\listoffigures
\newpage

%======================================================================


%% Title Page %%%%%%%%%%%%%%%%%%%%%%%%%%%%%%%%%%%%%%%%%%%%%%%
%% ==> Write your text here or include other files.

%% The simple version:
%\title{Címoldal}
%\author{Nyitrai Tamás}
%\date{} %%If commented, the current date is used.
%\maketitle

%% Inhaltsverzeichnis %%%%%%%%%%%%%%%%%%%%%%%%%%%%%%%%%%%%%%%
%\tableofcontents %Table of contents
%\cleardoublepage %The first chapter should start on an odd page.

\pagestyle{plain} %Now display headings: headings / fancy / ...



%% Chapters %%%%%%%%%%%%%%%%%%%%%%%%%%%%%%%%%%%%%%%%%%%%%%%%%
%% ==> Write your text here or include other files.

%\input{intro} %You need a file 'intro.tex' for this.


%%%%%%%%%%%%%%%%%%%%%%%%%%%%%%%%%%%%%%%%%%%%%%%%%%%%%%%%%%%%%

%1. fejezet
\chapter{Bevezetés}
\label{bev}
%1. fejezet
Az okos-telefonok térhódítása miatt már hazánkban is a lakosság fele rendelkezik valamilyen okos eszközzel, ez a szám pedig a jövőben egyre csak növekedni fog. A mindennapi élet megkönnyítésére rengeteg féle alkalmazás születik napról napra. Egyetlen eszközön olvashatunk újságot, tudakozódhatunk a közlekedésről, vehetünk ebédet magunknak, vagy unaloműzésként játszhatunk. Manapság minden korosztály talál kedvére való játékot, legyen szó akár ingyenes akár fizetős verzióról. Napjainkban igencsak elterjedtek az olyan játékok ahol a felhasználók interakciókba léphetnek egymással. Ezek túlnyomó többsége az úgynevezett micro-paymentekre alapszik, ahol a játékosok csekély összegekért cserébe előnyökhöz, könnyítésekhez juthatnak. A „free to play, pay to win” kifejezést azokra a játékokra szokták használni, ahol az előbb említett vásárlások nélkül képtelenség megnyerni a játékot, mert túlzottan befolyásolják a játékosok fejlődését.

Mivel egyre több időt töltünk ezen okos eszközök előtt, ezért egyre több ember életéből hiányzik az állandó testmozgás. Manapság szükség van különféle ösztönző módszerekre. Ezek többféle módon is megnyilvánulhatnak, léteznek különböző virtuális díjazások, sportolásért járó pontok, amik később tárgynyereményekre válthatóak, sőt vannak tényleges pénzzel való díjazások is.

Célom a két trend összekapcsolása oly módon, hogy a játékban a gyorsabb fejlődést nem pénzkifizetéssel, hanem sportolással lehet kiváltani.

A 2. fejezetben ismertetem az elkészülő játék hátterét, bemutatok különböző játéktípusokat.

A 3. fejezetben ismertetem a program tervezett funkcionalitását és követelményeit. Illetve kitérek a játék egy fontos alapelemére az általam választott dream-iso-droid játékmotorra.

A 4. fejezetben bemutatom az elkészült alkalmazást felhasználói szemmel.

Az 5. fejezetben részletesen kifejtem a megvalósított játék fő alkotóelemeit, és azok implementációját.

[maradék fejezet]

%2. fejezet
\chapter{Hasonló fejlesztések, szerepjátékokról általánosan}
\label{bem}
%2. fejezet

A videójátékok által kínált szórakozás napjainkban nagy népszerűségnek örvend, akár a konzolokra készült játékokról, akár a számítógépes platformra készültekről van szó.
Azonban nem volt mindig ilyen populáris, fejlődése a technológia előrehaladásával valósulhatott meg. 
A következő fejezetben bemutatom hogyan is változtak a videójátékok az idő múlásával. 
Ezt követően ismertetek néhány játékot, amik jelenleg a piacon vannak és hasonló célkitűzéssel születtek meg, mint az általam fejlesztett alkalmazás: az emberek mozgásra ösztönzése miatt. 
Végül pedig áttekintést adok az RPG-ről, mint játékstílusról. 
Célom, hogy betekintést adjak az említett műfajról, illetve hogy miért erre a műfajra esett a választásom a játék elkészítésekor. 


%2.1
\section{A videójátékok története}
\label{videojatektortenet}

% \Picture{A társasjáték}{2/1}{width=10cm}

Videójátékoknak nevezzük azokat a típusú játékokat, ahol a játékosok egy felhasználói felületen keresztül lépnek interakcióba a játékkal. 
Története az 1950-es évekre vezethető vissza, ebben az évtizedben kezdte el foglalkoztatni az embereket az az ötlet, hogy a szórakozás élményét elektromos eszközök segítségével érjék el. 
Az első videójáték egy asztalitenisz szimulátor volt, 1958-ban készítette el William Higinbotham és a „Tennis for Two” nevet viselte, mely \aref{fig_2/tennisfortwo}. ábrán látható. 

\Picture{Tennis for two - első videójáték}{2/tennisfortwo}{width=10cm}

A játék egy asztali analóg számítógépre készült el, és egy oszcilloszkópot használt a megjelenítésre.  
Az első olyan játék, amelyet már egy kontroller ősének nevezhető eszközzel irányítottak 1962-ben készült el, és a Spacewar nevet kapta. 
Itt két űrhajót irányíthattak a játékosok, a cél pedig a másik fél letorpedózása volt. 

A Spacewar változataként jelent meg a világ első pénzbedobós játéktermi gépe a Computer Space. 
Ennek ellenére ez a játék feledésbe merült, a népszerűséget az 1972-ben megjelent PONG játék szerezte meg. 
A játék annyira népszerű volt, hogy 3 év múlva az otthoni verziója is elkészült a játéknak. 
Ennek a sikernek köszönhető, hogy megindult az otthoni konzolok fejlesztése. 
Az emberek korábban játéktermekbe jártak ha szórakozni vágytak, viszont ettől kezdve rá tudták kötni a TV-re is otthon és onnan élvezni a játékot. 

\Picture{Donkey Kong játék egy jelenete}{2/donkey_kong}{width=7cm}

Ezt követően megindult a játéktermi gépek térhódítása a világon. 
Az Atari mellett a Sega és a Midway cégek is belekezdtek a saját játékgépeik fejlesztésébe. 
Ekkoriban mindenki be akart szállni a játékgyártásba, emiatt gyorsan sok videójátékot termeltek ki a vállalatok. 
Az 1980-as évek elején a sok silány minőségű játék miatt halottnak nyilvánították a videójáték piacot, mert a emberek nem kívántak rossz minőségű játékokat venni, ennek következtében csökkent a keresletük. 
Az első olyan videójátékot, ami történetet mesél el a Nintendo cégnek köszönhetjük. 
A játék az 1981-ben kiadott Donkey Kong volt, ahol a ma is híres Mario (akkori nevén Jumpman) karakterével kellett megmenteni a bajbajutott lányt, a játékból egy jelenet \aref{fig_2/donkey_kong} ábrán tekinthető meg. 

1985-től a videójáték piac újból felemelkedett a Nintendo Entertainment System nevű otthoni játékkonzolnak köszönhetően. 
Egymás után jelentek meg különböző vállalatok játékai, ezek között találhatóak voltak stratégiai, verekedős továbbá kalandozós játékok is. 
Az 1990-es évektől kezdve folyamatosan jelennek meg a felfejlesztett játékkonzolok. 
A hozzájuk készült játékok nem csak a fajtájukban térnek el, hanem különböző megjelenésükkel és irányítási rendszerükkel egyedivé varázsolják a játékélményt. 
A közeljövőben elérhetővé válik majd az a játékforma, amikor a játékos érzékeit becsapják a képek és a hangok, és teljesen úgy fogja érezni, mintha belecsöppent volna a játék világába. 

A játékok egy bizonyos fajtáját, a mobiljátékokat úgy definiálhatjuk, hogy hordozható telekommunikációs platformra készült játékok. 
Az első ilyen játék nem is igazán mobilra készült, hanem egy számológépen futott. 
A játék maga pedig egy Snake nevű program volt, ahol egy kígyót kellett irányítani és minél hosszabbra növeszteni azzal, hogy a képernyőn megjelent falatokat megetetjük vele. 
A Gameloft játékfejlesztő cég volt az első, aki nem volt mobiltelefongyártóhoz kötve, hanem az önálló játékfejlesztést tűzte ki céljául. 
A 2000-es évek elején megjelentek a színes kijelzővel rendelkező telefonok, amik új lehetőségeket biztosítottak a játékfejlesztőknek. 
Ekkoriban a játékok minőségét főleg a kijelző nagysága korlátozta. 
A játékok grafikai javulása miatt nőtt a méretük, amíg az első telefonos játékok 60 kilobájt körül mozogtak, addig 2007-re méretük elérte a 600 kilobájtot is. 

A következő fordulópontot az első okostelefon megjelenése jelentette, nemcsak mérete, de érintőképernyős kezelőfelülete miatt is. 
A mobiltelefonok fejlődésével a játékok is fejlődésnek indultak. 
A mai okostelefonra készült játékok egy része vetekszik a számítógépre készült játékok színvonalával. 
Léteznek olyan mobilok amelyek kifejezetten játékok gyakori használatára terveztek, illetve elérhető már olyan kontroller, amit mobileszközökre lehet csatlakoztatni és igazi konzolos játékélményt nyújt a felhasználóknak. 

\section{Sportjátékok}
\label{sportjatekok}

Az ötlet, hogy a játékokat és a sportokat össze lehet kapcsolni már egy ideje létezik, ezeket a játékokat "exergames" névvel illetik, a gyakorlat és a játék szavak összeolvasztásából. 
Léteznek konzolos játékok is amelyek a mozgáskövető rendszerük segítségével állapítják meg, hogy a játékos helyesen végzi e a gyakorlatokat. 
Továbbá léteznek mobilos alkalmazások is, ezek főleg ösztönzésre vagy sportolás közbeni szórakoztatásra fókuszálnak. 
Ezek közül mutatok be pár sikeresebb példát, amelyek jelenleg a piacon találhatóak. 

\subsection{Zombies, Run!}
\label{zombiesrun}
\Picture{A Zombies, Run! felülete}{2/zombiesrun}{width=14cm}
A Zombies, Run! egy népszerű futó alkalmazás, ahol a felhasználó egy zombi-apokalipszisbe csöppen bele túlélőként. 
A játék sikerét mutatja, hogy Android felületen fél millió felhasználó használja jelenleg \cite{zombiesrun}. 
A játékos feladata, hogy minél több zsákmányt szerezzen futás közben, amivel egy bázist kell folyamatosan fejlesztenie, hogy az ott állomásozó túlélőket biztonságban tudhassa. 
A megszerzett zsákmányokról és az építendő bázisról demonstráló képeket \aref{fig_2/zombiesrun} ábrán látható. 
A felhasználó kezdetben elindítja az alkalmazást, majd a futás közben elkezdődik a játék történetének mesélése. 
Ez nem folyamatos, a felhasználó tud közben zenét hallgatni. 
Futás közben találkozhatunk zombikkal, melyeket vagy a futás közben talált ellátmányért cserébe rázhatunk le, vagy egy rövid ideig 20\%-kal gyorsabban kell futnunk. 
Ez a folyamatos váltakozás a futás közben egyfajta intervallum edzésnek felel meg, melynek igen sok előnyös tulajdonsága van. 
Az alkalmazás önmagában egy sporttracker, ami szenzorok segítségével követi nyomon a felhasználó sportolási tevékenységét.

\subsection{Tep}
\label{tep}
\Picture{Tep játék felülete}{2/tep}{width=14cm}
Meg kell továbbá említeni a Tep-et, amely magyar fejlesztésű, a játékról pillanatfelvételeket \aref{fig_2/tep} ábrán láthatunk. 
A Tep szintén motivációs sport-nyomkövető alkalmazás, mely a valós teljesítmények után ad jutalmat a játékban. 
A játék stílusa a népszerű Tamagotchi játékhoz hasonló, azaz egy virtuális állatkát kell gondoznunk mindennaposan. 
A kapott jutalmakat beválthatjuk a virtuális állatunk részére különböző étel, ital és dekoratív elemre is. 
Ez az alkalmazás is különálló sport-nyomkövetőként működik, azaz nem külső forrásból szerzi be az adatokat, ugyanakkor össze lehet kötni hordozható eszközökkel, a Fitbit-tel és a Jawbone eszközökkel. 
Annak ellenére, hogy a játék motivációs célt szolgál, a felhasználó kevésbé van ösztönözve a sportolásra, ugyanis ha nem sportol folyamatosan, az egyetlen változás, ami bekövetkezik, hogy ha az állatkát "simogatjuk", akkor nem csóválja a farkát és éhezik. 

\subsection{Pokémon Go}
\label{pokemongo}
\Picture{Pokémon GO játékelemei}{2/pokemongoabra}{width=12cm}
A Pokémon Go az RPG játékok egy speciális fajtába az MMORPG-be tartozik. 
A játék kizárólagosan csak mobil eszközre készült abból az okból kifolyólag, hogy közvetett vagy közvetlen módon sportolásra vagy legalább mozgásra ösztönözze az embereket. 
Emiatt szükség volt arra, hogy az eszköz, amin játsszanak, mobilis legyen. 
Mindezek mellett a játék félig a valóságban félig pedig a virtuális világban játszódik. 
\Aref{fig_2/pokemongoabra}. ábra b) képén látható módon a pontos pozíciónkat megjeleníti a térképen, ami a valós világ útjaira, épületeire alapszik. 
Annyiban viszont eltér, hogy a játék egyes elemeit például a pokémonokat (kitalált állatszerű lények) a virtuális világban a térképre helyezi, majd a felhasználónak a való világban fizikailag oda kell jutnia hozzá, hogy elkaphassa, amely \aref{fig_2/pokemongoabra}. ábra a) képén látható. 

\section{Az RPG játékokról általánosságban}
\label{rpgaltalanos}

A következő alfejezetben az RPG-t (role playing game), vagyis szerepjátékot fogom bemutatni. 
Ezek a fajta játékok arra épülnek, hogy a felhasználó egy karakter "szerepébe” bújik bele, őt irányítva végzi el a feladatokat, kalandozik a világban. 
Eredete az ókorba vezethető vissza, elődjének tekinthetőek a különböző harci játékok amelyek az ütközetek szimulálására szolgáltak. 
Az első szerepjáték az 1974-ben megjelent Dungeons \& Dragons volt, ami hasonló szabályrendszerrel rendelkezett, mint a napjainkban megjelenő RPG-k. 
Előnyük, hogy sok ember számára elérhetőek, ugyanis a játékhoz dobókockákra, papírra és képzelőerőre van szükség. 
A mesélőnek kinevezett személy vezeti végig a kalandozáson a többi szereplőt. 
Minden játékoshoz tartozik egy karakter, aki fölött rendelkezhet, illetve a karakterlapján vezetheti a statisztikákat és jellemzőket, amint \aref{fig_2/dungeon_and_dragons_character_sheet}. ábrán is látható. 
Egy elvégzett feladat vagy küldetés után különböző jutalmakat kaphatnak a karakterek, amelyek fejlődésük során egyre erősebbek lesznek, emiatt pedig sikerül elérniük a játék elején kitűzött céljukat. 

\Picture{Dungeons \& Dragons karakterlap}{2/dungeon_and_dragons_character_sheet}{width=14cm}

Nem kellett sok idő ahhoz, hogy az RPG meghódítsa a számítógépes közeget is. 
Az 1970-es évek közepe után sorra jelentek meg a többfelhasználós kalandjátékok, amik a szerepjátékok szabályait követve nyújtottak szórakozási lehetőséget azoknak, akik rendelkeztek internetkapcsolattal. 
Ennek a mintájára napjainkban már nem csak webes felületen érhetőek el a hasonló típusú játékok, hanem az okostelefonok terjedésével, már mobil felületen is. 
A grafikai kártyák fejlődése következében az utóbbi két játékforma grafikai elemek felhasználásával szimulálja a különböző akciókat, amelyek a régi típusú szerepjátékban a képzeletre voltak bízva. 
Továbbá egy jelentős különbség még, hogy míg az eredeti szerepjátékokban a harcok kimenetele többnyire a szerencsén múlik, addig az online játékoknál különböző képletek és algoritmusok segítségével számolják ki, egy egy támadás mértékét. 
Feltehetőleg a komplexebb harcrendszer miatt alakult ki több változata a küzdelem lebonyolításának. 
Egy népszerű formája a "turn based" vagyis körökre osztott összecsapás. 
A játékos karaktere és az ellenfél felváltva támad, minden fél a saját körében dönt arról, hogy milyen cselekvést fog végrehajtani. 
Ez az akció lehet támadás, öngyógyítás vagy esetleg menekülési kísérlet. 
Ezen kívül vannak szimulált harcot implementáló játékok, ahol a harc kimenetele időközben nem befolyásolható.
Itt az összecsapás az ellenfél és a saját karakter tulajdonságpontjainak a felhasználásával kerül kiszámításra. 
A játékos a végeredményt látja csak, hogy sikerült e legyőzni az ellenséget vagy sem. 

Továbbá különbséget tehetünk abban, hogy az online felületen játszható játékok hosszú távon tudnak szórakozási lehetőséget biztosítani, nem szükséges a játékosok folyamatos jelenléte. 
Az online szerepjátékokban jellemzően nincs kitűzött végcél, a felhasználók kalandoznak, fejlődnek és egyre erősödő ellenfeleket győznek le. 
Ha a játékban nincsenek maximálisan elérhető értékek definiálva, akkor  csak a játékos kitartása és eltökéltsége szab határt a játék végének. 
A cél egy olyan játék megvalósítása volt, ami hosszú távon képes ösztönözni a felhasználót a sportolásra.

\newpage

%3. fejezet
\chapter{Követelmények, technológiák}
\label{fejlesztes}
%3. fejezet

A fejezetben szó esik a szoftverrel szemben támasztott követelményekről, valamint a felhasznált programok technológiák kerülnek bemutatásra. 

%3.1
\section{Követelmények}
\label{kovetelmenyek}

Az alkalmazással szemben különféle követelményeket támasztok, melyeknek mindenképp meg kell felelnie, melyek a következőek: 

\begin{itemize}
	\item Motiválás: 
	\\
	Első és legfontosabb követelmény, hogy képes legyen az embereket ösztönöznie a sportolásra. 
	Ezt minél érdekesebb és izgalmasabb játékmechanikai elemekkel kívánom elérni. 
	\item Kiterjeszthetőség: 
	\\
	Az elkészítendő játék nem különálló tracker alkalmazásként fog működni, önmagában nem lesz képes mérni a sporttevékenységeket. 
	Minden esetben más tracker szolgáltatások által elmentett tevékenységet fog lekérni az alkalmazás. 
	Emiatt fontos, hogy minél több alkalmazástól tudjon lementett sporttevékenységeket lekérni. 
	Amiatt is fontos lenne minél több tracker alkalmazás támogatása, mivel azon felhasználók, akik már régebb óta sportolnak, ne kelljen a számukra bevált sport nyomkövető szolgáltatást lecserélni. 
	\item Érdeklődés fenntartás 
	\\
	Miután a játék felkeltette a felhasználó figyelmét, el kell érni, hogy tovább játsszon vele. 
	Ahogy a korábbiakban említettem, a játékmechanikai elemekkel és különböző kihívásokkal szeretném megvalósítani, melyeket lejjebb fejtek ki bővebben. 

	\item Gyors, sok eszközön, ezt majd ki kell fejteni.... 
	\\
	Kifejtés.. 
\end{itemize}


%3.1.1
\section*{Funkcionális követelmények}
\label{funkckovetelmenyeks}

Az alábbiakban a főbb funkcionális követelmények kerülnek bemutatásra. 

\begin{itemize}
	\item 
	A játéknak képesnek kell lennie csatlakozni sport-nyomkövető alkalmazásokhoz. 
	Lehetőleg minél több alkalmazást kell támogatnia a játéknak, mivel így nagyobb lehet az elérhető potenciális játékos-közösség is. 
	\item 
	A minél nagyobb számú támogatottság elérése érdekében az alkalmazásnak mindenképp támogatnia kell az OAuth szabványt. 
	A legtöbb, ha nem minden sport tracker alkalmazás ezt a szabványt használja a külső alkalmazásokkal való kapcsolódásra. 
	Amikor csatlakozni szeretnénk az adott profilunkhoz külső alkalmazásból, meg kell adni az engedélyt, hogy az alkalmazás mely adatainkhoz férjen hozzá. 
	\item 
	Csatlakozás után az alkalmazásnak le kell töltenie a felhasználó legújabb sport tevékenységeit. 
	Erőforrás takarékosság szempontjából először meg kell bizonyosodni, hogy van-e új tevékenység. 
	Törekedni kell, hogy a felhasználóhoz tartozó összes adatot csak az első csatlakozás alkalmával, vagy más eszközön való bejelentkezés esetén töltsük le. 
	\item 
	Ha a felhasználó nem csatlakozik más sport tracker alkalmazáshoz, akkor bejelentkezés esetén kell megbizonyosodni, hogy van-e csatlakoztatott sport trackerek esetén történt-e új tevékenység felvitel. 
	Amennyiben igen, úgy csak ezeket az új tevékenységeket kell letölteni.
	\item 
	A letöltött adatokat az Androidos eszközökön kell tárolni. 
	Erre azért van szükség, hogy a későbbiekben a régebbi eseményeket, vagy a már jutalmazott tevékenységekért ne adjunk újra bónuszt. 
	Ennél a pontnál szembe kell nézni a ténnyel, hogy a különböző sport trackerek bizonyos adatokat másként tárolnak, vagy teljesen hiányoznak. 
	Emiatt létre kell hozni egy olyan általános adatbázis táblát, amelyben minden olyan adatot tárolunk, amelyekért jutalmat akarunk osztani a játékos számára. 
	A későbbiekben támogatottságot nyerő sport trackereknek így valamilyen módon szolgáltatnia kell legalább azokat az információkat, amik ebben az adattáblában kapnak helyet. 
	Az adattípusuk különbözhet, és amennyiben egy bizonyos adatot nem szolgáltat, de más adatokból származtatni lehet, úgy az nem okozhat akadályt.
	\item 
	Bejelentkezés után az újonnan letöltött adatok alapján a felhasználó staminát (kitartást) kap. 
	Egy játékosnak maximum 100 staminája lehet. 
	A kapott stamina mennyisége összhangban kell lennie ezzel a maximális értékkel, a játékos szintjével, és a tevékenységben szereplő adatok nagyságával. 
	Azaz az alacsony és magas szintű felhasználóknak is egyaránt élvezetesnek kell maradnia a játéknak, nem szabad se túl sokat, se túl keveset kapni. 
	Túl sok stamina esetén nagyon könnyen haladhatna a felhasználó a játékban, így egy idő után beleunna, túl kevés esetén viszont a folyamatosan túl nagy kihívást jelentő és csak nagy megerőltetést jelentő tevékenységek szintén ugyanezt a hatást érnék el.
	\item 
	A jutalomként megkapott staminát a felhasználó a játékosa fejlődésére használhatja fel különböző módokon. 
	Az egyik ilyen mód a világban való "barangolás", ami közben szörnyek támadhatnak a játékosra, amelyeket legyőzve játékbeli pénzt és tapasztalati pontot kap a játékos. 
	A másik mód küldetések vállalása, amelyet a felhasználónak kell ténylegesen sportolva teljesíteni, és csak a teljesítése után kapja meg az érte járó játékbeli jutalmat.
	\item 
	A játékos ezen kívül rendelkeznie kell tulajdonságokkal is, melyek a szörnyek elleni csatában segíthetnek számára. 
	Tulajdonságot növeli szintlépéssel vagy valamilyen kirívó sportteljesítményért cserébe lenne érdemes megengedni.
	\item 
	További tárgyakat is érdemes lenne megvalósítani a játékos számára, melyek védelemmel vagy támadóerővel növelhetnék a játékos erejét.
\end{itemize}


\section{Felhasznált technológiák}
\label{felhtechnologia}

\section*{Játékmotorok}
\label{jatekmotor}

Játékmotornak nevezzük a játékok - legyen az akár számítógépre vagy konzolra készült – azon részét, amely a program alapjául szolgáló technológiát adja. 
Szerepe, hogy megkönnyítse a fejlesztést illetve segítségével több platformon is futtatható lesz a játék.

A fejlesztés megkezdése előtt több fajta játékmotort is megvizsgáltam abból a célból, hogy kiválasszam a legmegfelelőbbet a diplomamunkám elkészítéséhez. 

A fő szempontom az volt, hogy ingyenesen elérhető legyen, illetve illeszkedjen a választott játéktípus játékmenetéhez.

A két legnépszerűbb motorral kezdtem az ismerkedést, a Unity és az Unreal engine-ekkel. 
Mivel a programomat Android platformon terveztem elkészíteni, amit Java nyelven kell implementálni, ezek a motorok pedig a C++ nyelvet támogatják, így nem lehet közvetlenül Java nyelven használni őket ezért hamar kiestek. 
Méretük alapján túl nagynak is bizonyultak volna egy ilyen kisebb méretű projekthez. 
A következő játékmotor, amit megvizsgáltam a HexEngine volt, amit Szabó László készített el MSc diplomamunkájaként. 
Ez a motor kifejezetten körökre osztott játékokra lett kifejlesztve, amivel az általam írt játék is rendelkezik, viszont a játéktér hatszögű blokkokra van osztva, amivel megbonyolította volna a közlekedést a játékon belül.

A választásom így Hollósi Tamás által készített dream-iso-droid játékmotorra esett, amit témavezetőm ismertetett meg velem. 
Mivel készítője elérhető közelségben volt, ezért könnyebben sikerült megismerkednem a motor nyújtotta funkciókkal.

\Picture{Játékmotorok összehasonlítása}{3/2}{width=10cm}

A dream-iso-droid egy olyan speciális játékmotor, ami kifejezetten Android platformra készült és a két dimenziós izometrikus nézetet támogatja. 
Ez a két funkciója pontosan megfelelt az elvárásaimnak, amit a játékmotor felé támasztottam, aminek segítségével fejleszteni szerettem volna az RPG játékomat.







%% <== End of hints
%%%%%%%%%%%%%%%%%%%%%%%%%%%%%%%%%%%%%%%%%%%%%%%%%%%%%%%%%%%%%


%%%%%%%%%%%%%%%%%%%%%%%%%%%%%%%%%%%%%%%%%%%%%%%%%%%%%%%%%%%%%
%% BIBLIOGRAPHY AND OTHER LISTS
%%%%%%%%%%%%%%%%%%%%%%%%%%%%%%%%%%%%%%%%%%%%%%%%%%%%%%%%%%%%%
%% A small distance to the other stuff in the table of contents (toc)
\addtocontents{toc}{\protect\vspace*{\baselineskip}}


\bibliographystyle{mybibstyle}
\bibliography{cite}

%% The List of Figures
%\clearpage
%\addcontentsline{toc}{chapter}{List of Figures}
%\listoffigures

%% The List of Tables
%\clearpage
%\addcontentsline{toc}{chapter}{List of Tables}
%\listoftables


\newpage

\Large
\begin{center}
	\textbf{MELLÉKLET}
\end{center}
\normalsize
\noindent
A mellékelt CD könyvtárszerkezete


% \begin{itemize}
%    \item \textbf{Dokumentum}
%    \begin{itemize}
%        \item \textbf{Forrás} - A szakdolgozat szerkeszthető formátumban
%        \item \textbf{Hivatkozások} - A szakdolgozatban lévő internetes hivatkozások letöltve
%        \item szakdolgozat.pdf
%    \end{itemize}
%    \item \textbf{Forrás} - A program forrásállománya
%    \begin{itemize}
%        \item \textbf{HexEngine}
%        \item \textbf{HexEngine-android}
%        \item \textbf{HexEngine-desktop}
%    \end{itemize}
%    \item \textbf{Program} - A futtatható program
%    \begin{itemize}
%        \item \textbf{bin} - A program grafikai és konfigurációs fájljai ami szükségesek az indításhoz
%        \item \textbf{hav} - A program a mentéseket tárolja itt
%        \item \textbf{java\_telepito} - A Java környezet telepítői
%        \item \textbf{libs} - Futtatáshoz kapcsolódó java fájlok
%        \item starter.jar
%    \end{itemize}
% \end{itemize}





%%%%%%%%%%%%%%%%%%%%%%%%%%%%%%%%%%%%%%%%%%%%%%%%%%%%%%%%%%%%%
%% APPENDICES
%%%%%%%%%%%%%%%%%%%%%%%%%%%%%%%%%%%%%%%%%%%%%%%%%%%%%%%%%%%%%
\appendix
%% ==> Write your text here or include other files.

%\input{FileName} %You need a file 'FileName.tex' for this.


\end{document}

