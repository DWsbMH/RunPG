%% Based on a TeXnicCenter-Template by Tino Weinkauf.
%%%%%%%%%%%%%%%%%%%%%%%%%%%%%%%%%%%%%%%%%%%%%%%%%%%%%%%%%%%%%


%%%%%%%%%%%%%%%%%%%%%%%%%%%%%%%%%%%%%%%%%%%%%%%%%%%%%%%%%%%%%
%% HEADER
%%%%%%%%%%%%%%%%%%%%%%%%%%%%%%%%%%%%%%%%%%%%%%%%%%%%%%%%%%%%%
\documentclass[a4paper,oneside,10pt]{report}
% Alternative Options:
%	Paper Size: a4paper / a5paper / b5paper / letterpaper / legalpaper / executivepaper
% Duplex: oneside / twoside
% Base Font Size: 10pt / 11pt / 12pt

\usepackage{hegyhati}


%%%%%%%%%%%%%%%%%%%%%%%%%%
\usepackage{t1enc} 
\usepackage{lmodern} 


%%%%%%%%%%%%%%%%%%%%%%%%%%%%%%%%%%%%%%%%%%%%%%%%%%%%%%%%%%%%%
%% DOCUMENT
%%%%%%%%%%%%%%%%%%%%%%%%%%%%%%%%%%%%%%%%%%%%%%%%%%%%%%%%%%%%%
\begin{document}



\begin{titlepage}
\begin{center}
\Large
Pannon Egyetem

\vspace{10mm}
Műszaki Informatikai Kar

\vspace{10mm}
Rendszer- és Számítástudományi Tanszék

\vspace{10mm}
Mérnökinformatikus MSc

\vspace{40mm}
\huge
DIPLOMAMUNKA

\vspace{10mm}
\LARGE
"Free to play, run to win" játék Androidra

\vspace{10mm}
\Large
Nyitrai Tamás

\vspace{40mm}
Témavezető: Dr. Hegyháti Máté

\vspace{10mm}
2016
\normalsize
\end{center}
\end{titlepage}



\pagestyle{empty} %No headings for the first pages.



\newpage
\Large
\begin{center}
	\textbf{KÖSZÖNETNYILVÁNÍTÁS}
\end{center}
\normalsize
\noindent
Meg szeretném köszönni édesanyámnak támogatását és folyamatos bátorítását. Továbbiakban köszönettel tartozom témavezetőmnek, Dr. Hegyháti Máténak, aki ötleteivel és tanácsaival végig a helyes úton tartott.

Továbbá szeretném megköszönni Hollósi Tamásnak a dream-iso-android elkészítőjének a segítségét, akihez bátran fordulhattam, ha bármi kérdésem volt a játékmotorral kapcsolatban. Valamint köszönöm Böröndi Evelinnek a grafikák elkészítésében vállalt segítségét.
\newpage
\Large
\begin{center}
	\textbf{TARTALMI ÖSSZEFOGLALÓ}
\end{center}
\normalsize
\noindent

Az utóbbi évtized során az okoseszközök népszerűsége évről-évre folyamatosan növekedett és ennek következtében már hazánkban is a lakosság több, mint fele rendelkezik ilyen eszközzel. 
Egy ilyen okoseszközt rengeteg módon lehet felhasználni, például újságot olvashatunk rajta, bevásárolhatunk ezen keresztül, böngészhetünk menetrendek közt vagy akár játékokkal is elüthetjük az időnket. 
Az okoseszközök térhódításával szinte párhuzamosan terjedtek el az olyan játékok amelyek a micro-paymentekre alapszanak, vagyis az olyan játékok ahol a játékos kis összegű vásárlással könnyítéshez, előnyhöz juthat, ezekre a játékokra szokás használni a "free to play, pay to win” kifejezést.

A rendszeres testmozgás egyre több ember napjaiból hiányzik, hisz kényelmi okokból miatt az autós vagy tömegközlekedést választják. 
Az eszközök eltöltött minél több idő csak súlyosbítja a rendszeres testmozgást hiányát. 
Sok ember számára nincs elegendő motiváló hatása annak, hogy a rendszeres testmozgás egészségügyi szempontból fontos része a mindennapoknak. 
A hiányzó motiváció betöltésére különböző lehetőségek tárulkoznak az emberek felé, mint amilyen a virtuális díjazás, a sportolásért járó pontok amiket későbbiekben tárgynyereményekre lehet cserélni vagy akár a készpénzzel való díjazás. 

Elkészítettem egy Androidos szerepjátékot a két trend összekapcsolása érdekében. 
A játékon belül a felhasználó a főszereplő bőrébe bújva kell megküzdeni az ellenségekkel, így képes a folyamatos fejlődésre. 
A karakter csak limitált távolságot tud megtenni, és hogy tovább tudjon haladni, a felhasználónak testmozgást kell végeznie, ezáltal ösztönözve őt. 

\textbf{Kulcsszavak:} dream-iso-droid, Android, testmozgás, szerepjáték

\newpage

\Large
\begin{center}
	\textbf{ABSTRACT}
\end{center}
\normalsize
\noindent
Angol tartalmi összefoglaló...

\textbf{Keywords:} dream-iso-droid, Android, sports, RPG game
\tableofcontents
\newpage
\listoffigures
\newpage

%======================================================================


%% Title Page %%%%%%%%%%%%%%%%%%%%%%%%%%%%%%%%%%%%%%%%%%%%%%%
%% ==> Write your text here or include other files.

%% The simple version:
%\title{Címoldal}
%\author{Nyitrai Tamás}
%\date{} %%If commented, the current date is used.
%\maketitle

%% Inhaltsverzeichnis %%%%%%%%%%%%%%%%%%%%%%%%%%%%%%%%%%%%%%%
%\tableofcontents %Table of contents
%\cleardoublepage %The first chapter should start on an odd page.

\pagestyle{plain} %Now display headings: headings / fancy / ...

\chapter{Bevezetés}
\label{bev}
%1. fejezet
Az okos-telefonok térhódítása miatt már hazánkban is a lakosság fele rendelkezik valamilyen okos eszközzel, ez a szám pedig a jövőben egyre csak növekedni fog. A mindennapi élet megkönnyítésére rengeteg féle alkalmazás születik napról napra. Egyetlen eszközön olvashatunk újságot, tudakozódhatunk a közlekedésről, vehetünk ebédet magunknak, vagy unaloműzésként játszhatunk. Manapság minden korosztály talál kedvére való játékot, legyen szó akár ingyenes akár fizetős verzióról. Napjainkban igencsak elterjedtek az olyan játékok ahol a felhasználók interakciókba léphetnek egymással. Ezek túlnyomó többsége az úgynevezett micro-paymentekre alapszik, ahol a játékosok csekély összegekért cserébe előnyökhöz, könnyítésekhez juthatnak. A „free to play, pay to win” kifejezést azokra a játékokra szokták használni, ahol az előbb említett vásárlások nélkül képtelenség megnyerni a játékot, mert túlzottan befolyásolják a játékosok fejlődését.

Mivel egyre több időt töltünk ezen okos eszközök előtt, ezért egyre több ember életéből hiányzik az állandó testmozgás. Manapság szükség van különféle ösztönző módszerekre. Ezek többféle módon is megnyilvánulhatnak, léteznek különböző virtuális díjazások, sportolásért járó pontok, amik később tárgynyereményekre válthatóak, sőt vannak tényleges pénzzel való díjazások is.

Célom a két trend összekapcsolása oly módon, hogy a játékban a gyorsabb fejlődést nem pénzkifizetéssel, hanem sportolással lehet kiváltani.

A 2. fejezetben ismertetem az elkészülő játék hátterét, bemutatok különböző játéktípusokat.

A 3. fejezetben ismertetem a program tervezett funkcionalitását és követelményeit. Illetve kitérek a játék egy fontos alapelemére az általam választott dream-iso-droid játékmotorra.

A 4. fejezetben bemutatom az elkészült alkalmazást felhasználói szemmel.

Az 5. fejezetben részletesen kifejtem a megvalósított játék fő alkotóelemeit, és azok implementációját.

[maradék fejezet]

\chapter{Hasonló fejlesztések, szerepjátékokról általánosan}
\label{bem}
%2. fejezet

A videójátékok által kínált szórakozás napjainkban nagy népszerűségnek örvend, akár a konzolokra készült játékokról, akár a számítógépes platformra készültekről van szó.
Azonban nem volt mindig ilyen populáris, fejlődése a technológia előrehaladásával valósulhatott meg. 
A következő fejezetben bemutatom hogyan is változtak a videójátékok az idő múlásával. 
Ezt követően ismertetek néhány játékot, amik jelenleg a piacon vannak és hasonló célkitűzéssel születtek meg, mint az általam fejlesztett alkalmazás: az emberek mozgásra ösztönzése miatt. 
Végül pedig áttekintést adok az RPG-ről, mint játékstílusról. 
Célom, hogy betekintést adjak az említett műfajról, illetve hogy miért erre a műfajra esett a választásom a játék elkészítésekor. 


%2.1
\section{A videójátékok története}
\label{videojatektortenet}

% \Picture{A társasjáték}{2/1}{width=10cm}

Videójátékoknak nevezzük azokat a típusú játékokat, ahol a játékosok egy felhasználói felületen keresztül lépnek interakcióba a játékkal. 
Története az 1950-es évekre vezethető vissza, ebben az évtizedben kezdte el foglalkoztatni az embereket az az ötlet, hogy a szórakozás élményét elektromos eszközök segítségével érjék el. 
Az első videójáték egy asztalitenisz szimulátor volt, 1958-ban készítette el William Higinbotham és a „Tennis for Two” nevet viselte, mely \aref{fig_2/tennisfortwo}. ábrán látható. 

\Picture{Tennis for two - első videójáték}{2/tennisfortwo}{width=10cm}

A játék egy asztali analóg számítógépre készült el, és egy oszcilloszkópot használt a megjelenítésre.  
Az első olyan játék, amelyet már egy kontroller ősének nevezhető eszközzel irányítottak 1962-ben készült el, és a Spacewar nevet kapta. 
Itt két űrhajót irányíthattak a játékosok, a cél pedig a másik fél letorpedózása volt. 

A Spacewar változataként jelent meg a világ első pénzbedobós játéktermi gépe a Computer Space. 
Ennek ellenére ez a játék feledésbe merült, a népszerűséget az 1972-ben megjelent PONG játék szerezte meg. 
A játék annyira népszerű volt, hogy 3 év múlva az otthoni verziója is elkészült a játéknak. 
Ennek a sikernek köszönhető, hogy megindult az otthoni konzolok fejlesztése. 
Az emberek korábban játéktermekbe jártak ha szórakozni vágytak, viszont ettől kezdve rá tudták kötni a TV-re is otthon és onnan élvezni a játékot. 

\Picture{Donkey Kong játék egy jelenete}{2/donkey_kong}{width=7cm}

Ezt követően megindult a játéktermi gépek térhódítása a világon. 
Az Atari mellett a Sega és a Midway cégek is belekezdtek a saját játékgépeik fejlesztésébe. 
Ekkoriban mindenki be akart szállni a játékgyártásba, emiatt gyorsan sok videójátékot termeltek ki a vállalatok. 
Az 1980-as évek elején a sok silány minőségű játék miatt halottnak nyilvánították a videójáték piacot, mert a emberek nem kívántak rossz minőségű játékokat venni, ennek következtében csökkent a keresletük. 
Az első olyan videójátékot, ami történetet mesél el a Nintendo cégnek köszönhetjük. 
A játék az 1981-ben kiadott Donkey Kong volt, ahol a ma is híres Mario (akkori nevén Jumpman) karakterével kellett megmenteni a bajbajutott lányt, a játékból egy jelenet \aref{fig_2/donkey_kong} ábrán tekinthető meg. 

1985-től a videójáték piac újból felemelkedett a Nintendo Entertainment System nevű otthoni játékkonzolnak köszönhetően. 
Egymás után jelentek meg különböző vállalatok játékai, ezek között találhatóak voltak stratégiai, verekedős továbbá kalandozós játékok is. 
Az 1990-es évektől kezdve folyamatosan jelennek meg a felfejlesztett játékkonzolok. 
A hozzájuk készült játékok nem csak a fajtájukban térnek el, hanem különböző megjelenésükkel és irányítási rendszerükkel egyedivé varázsolják a játékélményt. 
A közeljövőben elérhetővé válik majd az a játékforma, amikor a játékos érzékeit becsapják a képek és a hangok, és teljesen úgy fogja érezni, mintha belecsöppent volna a játék világába. 

A játékok egy bizonyos fajtáját, a mobiljátékokat úgy definiálhatjuk, hogy hordozható telekommunikációs platformra készült játékok. 
Az első ilyen játék nem is igazán mobilra készült, hanem egy számológépen futott. 
A játék maga pedig egy Snake nevű program volt, ahol egy kígyót kellett irányítani és minél hosszabbra növeszteni azzal, hogy a képernyőn megjelent falatokat megetetjük vele. 
A Gameloft játékfejlesztő cég volt az első, aki nem volt mobiltelefongyártóhoz kötve, hanem az önálló játékfejlesztést tűzte ki céljául. 
A 2000-es évek elején megjelentek a színes kijelzővel rendelkező telefonok, amik új lehetőségeket biztosítottak a játékfejlesztőknek. 
Ekkoriban a játékok minőségét főleg a kijelző nagysága korlátozta. 
A játékok grafikai javulása miatt nőtt a méretük, amíg az első telefonos játékok 60 kilobájt körül mozogtak, addig 2007-re méretük elérte a 600 kilobájtot is. 

A következő fordulópontot az első okostelefon megjelenése jelentette, nemcsak mérete, de érintőképernyős kezelőfelülete miatt is. 
A mobiltelefonok fejlődésével a játékok is fejlődésnek indultak. 
A mai okostelefonra készült játékok egy része vetekszik a számítógépre készült játékok színvonalával. 
Léteznek olyan mobilok amelyek kifejezetten játékok gyakori használatára terveztek, illetve elérhető már olyan kontroller, amit mobileszközökre lehet csatlakoztatni és igazi konzolos játékélményt nyújt a felhasználóknak. 

\section{Sportjátékok}
\label{sportjatekok}

Az ötlet, hogy a játékokat és a sportokat össze lehet kapcsolni már egy ideje létezik, ezeket a játékokat "exergames" névvel illetik, a gyakorlat és a játék szavak összeolvasztásából. 
Léteznek konzolos játékok is amelyek a mozgáskövető rendszerük segítségével állapítják meg, hogy a játékos helyesen végzi e a gyakorlatokat. 
Továbbá léteznek mobilos alkalmazások is, ezek főleg ösztönzésre vagy sportolás közbeni szórakoztatásra fókuszálnak. 
Ezek közül mutatok be pár sikeresebb példát, amelyek jelenleg a piacon találhatóak. 

\subsection{Zombies, Run!}
\label{zombiesrun}
\Picture{A Zombies, Run! felülete}{2/zombiesrun}{width=14cm}
A Zombies, Run! egy népszerű futó alkalmazás, ahol a felhasználó egy zombi-apokalipszisbe csöppen bele túlélőként. 
A játék sikerét mutatja, hogy Android felületen fél millió felhasználó használja jelenleg \cite{zombiesrun}. 
A játékos feladata, hogy minél több zsákmányt szerezzen futás közben, amivel egy bázist kell folyamatosan fejlesztenie, hogy az ott állomásozó túlélőket biztonságban tudhassa. 
A megszerzett zsákmányokról és az építendő bázisról demonstráló képeket \aref{fig_2/zombiesrun} ábrán látható. 
A felhasználó kezdetben elindítja az alkalmazást, majd a futás közben elkezdődik a játék történetének mesélése. 
Ez nem folyamatos, a felhasználó tud közben zenét hallgatni. 
Futás közben találkozhatunk zombikkal, melyeket vagy a futás közben talált ellátmányért cserébe rázhatunk le, vagy egy rövid ideig 20\%-kal gyorsabban kell futnunk. 
Ez a folyamatos váltakozás a futás közben egyfajta intervallum edzésnek felel meg, melynek igen sok előnyös tulajdonsága van. 
Az alkalmazás önmagában egy sporttracker, ami szenzorok segítségével követi nyomon a felhasználó sportolási tevékenységét.

\subsection{Tep}
\label{tep}
\Picture{Tep játék felülete}{2/tep}{width=14cm}
Meg kell továbbá említeni a Tep-et, amely magyar fejlesztésű, a játékról pillanatfelvételeket \aref{fig_2/tep} ábrán láthatunk. 
A Tep szintén motivációs sport-nyomkövető alkalmazás, mely a valós teljesítmények után ad jutalmat a játékban. 
A játék stílusa a népszerű Tamagotchi játékhoz hasonló, azaz egy virtuális állatkát kell gondoznunk mindennaposan. 
A kapott jutalmakat beválthatjuk a virtuális állatunk részére különböző étel, ital és dekoratív elemre is. 
Ez az alkalmazás is különálló sport-nyomkövetőként működik, azaz nem külső forrásból szerzi be az adatokat, ugyanakkor össze lehet kötni hordozható eszközökkel, a Fitbit-tel és a Jawbone eszközökkel. 
Annak ellenére, hogy a játék motivációs célt szolgál, a felhasználó kevésbé van ösztönözve a sportolásra, ugyanis ha nem sportol folyamatosan, az egyetlen változás, ami bekövetkezik, hogy ha az állatkát "simogatjuk", akkor nem csóválja a farkát és éhezik. 

\subsection{Pokémon Go}
\label{pokemongo}
\Picture{Pokémon GO játékelemei}{2/pokemongoabra}{width=12cm}
A Pokémon Go az RPG játékok egy speciális fajtába az MMORPG-be tartozik. 
A játék kizárólagosan csak mobil eszközre készült abból az okból kifolyólag, hogy közvetett vagy közvetlen módon sportolásra vagy legalább mozgásra ösztönözze az embereket. 
Emiatt szükség volt arra, hogy az eszköz, amin játsszanak, mobilis legyen. 
Mindezek mellett a játék félig a valóságban félig pedig a virtuális világban játszódik. 
\Aref{fig_2/pokemongoabra}. ábra b) képén látható módon a pontos pozíciónkat megjeleníti a térképen, ami a valós világ útjaira, épületeire alapszik. 
Annyiban viszont eltér, hogy a játék egyes elemeit például a pokémonokat (kitalált állatszerű lények) a virtuális világban a térképre helyezi, majd a felhasználónak a való világban fizikailag oda kell jutnia hozzá, hogy elkaphassa, amely \aref{fig_2/pokemongoabra}. ábra a) képén látható. 

\section{Az RPG játékokról általánosságban}
\label{rpgaltalanos}

A következő alfejezetben az RPG-t (role playing game), vagyis szerepjátékot fogom bemutatni. 
Ezek a fajta játékok arra épülnek, hogy a felhasználó egy karakter "szerepébe” bújik bele, őt irányítva végzi el a feladatokat, kalandozik a világban. 
Eredete az ókorba vezethető vissza, elődjének tekinthetőek a különböző harci játékok amelyek az ütközetek szimulálására szolgáltak. 
Az első szerepjáték az 1974-ben megjelent Dungeons \& Dragons volt, ami hasonló szabályrendszerrel rendelkezett, mint a napjainkban megjelenő RPG-k. 
Előnyük, hogy sok ember számára elérhetőek, ugyanis a játékhoz dobókockákra, papírra és képzelőerőre van szükség. 
A mesélőnek kinevezett személy vezeti végig a kalandozáson a többi szereplőt. 
Minden játékoshoz tartozik egy karakter, aki fölött rendelkezhet, illetve a karakterlapján vezetheti a statisztikákat és jellemzőket, amint \aref{fig_2/dungeon_and_dragons_character_sheet}. ábrán is látható. 
Egy elvégzett feladat vagy küldetés után különböző jutalmakat kaphatnak a karakterek, amelyek fejlődésük során egyre erősebbek lesznek, emiatt pedig sikerül elérniük a játék elején kitűzött céljukat. 

\Picture{Dungeons \& Dragons karakterlap}{2/dungeon_and_dragons_character_sheet}{width=14cm}

Nem kellett sok idő ahhoz, hogy az RPG meghódítsa a számítógépes közeget is. 
Az 1970-es évek közepe után sorra jelentek meg a többfelhasználós kalandjátékok, amik a szerepjátékok szabályait követve nyújtottak szórakozási lehetőséget azoknak, akik rendelkeztek internetkapcsolattal. 
Ennek a mintájára napjainkban már nem csak webes felületen érhetőek el a hasonló típusú játékok, hanem az okostelefonok terjedésével, már mobil felületen is. 
A grafikai kártyák fejlődése következében az utóbbi két játékforma grafikai elemek felhasználásával szimulálja a különböző akciókat, amelyek a régi típusú szerepjátékban a képzeletre voltak bízva. 
Továbbá egy jelentős különbség még, hogy míg az eredeti szerepjátékokban a harcok kimenetele többnyire a szerencsén múlik, addig az online játékoknál különböző képletek és algoritmusok segítségével számolják ki, egy egy támadás mértékét. 
Feltehetőleg a komplexebb harcrendszer miatt alakult ki több változata a küzdelem lebonyolításának. 
Egy népszerű formája a "turn based" vagyis körökre osztott összecsapás. 
A játékos karaktere és az ellenfél felváltva támad, minden fél a saját körében dönt arról, hogy milyen cselekvést fog végrehajtani. 
Ez az akció lehet támadás, öngyógyítás vagy esetleg menekülési kísérlet. 
Ezen kívül vannak szimulált harcot implementáló játékok, ahol a harc kimenetele időközben nem befolyásolható.
Itt az összecsapás az ellenfél és a saját karakter tulajdonságpontjainak a felhasználásával kerül kiszámításra. 
A játékos a végeredményt látja csak, hogy sikerült e legyőzni az ellenséget vagy sem. 

Továbbá különbséget tehetünk abban, hogy az online felületen játszható játékok hosszú távon tudnak szórakozási lehetőséget biztosítani, nem szükséges a játékosok folyamatos jelenléte. 
Az online szerepjátékokban jellemzően nincs kitűzött végcél, a felhasználók kalandoznak, fejlődnek és egyre erősödő ellenfeleket győznek le. 
Ha a játékban nincsenek maximálisan elérhető értékek definiálva, akkor  csak a játékos kitartása és eltökéltsége szab határt a játék végének. 
A cél egy olyan játék megvalósítása volt, ami hosszú távon képes ösztönözni a felhasználót a sportolásra.

\newpage

\chapter{Követelmények, technológiák}
\label{kovetelmeny}
%3. fejezet

A fejezetben szó esik a szoftverrel szemben támasztott követelményekről, valamint a felhasznált programok technológiák kerülnek bemutatásra. 

%3.1
\section{Követelmények}
\label{kovetelmenyek}

Az alkalmazással szemben különféle követelményeket támasztottam, melyeknek mindenképp meg kellett felelnie, melyek a következőek: 

\begin{description}
	\item [Kiterjeszthetőség] 
    A játék alapköve, hogy a különböző testmozgásokat különböző pozitív jutalmakkal díjazza. 
	A sportolási tevékenységeket különböző sport-trackerekkel lehet mérni. 
	Az alkalmazásban ezt kétféle módszerrel lehet megvalósítani, egy hasonló működésű modult hozok létre, amely különböző szenzorok segítségével méri a sporttevékenységeket (GPS, gyorsulásmérő) vagy már meglévő szolgáltatásoktól szerezzük be ezeket az adatok. 
	Az utóbbi megoldás előnye, hogy a népszerűbb sport-tracker alkalmazások felhasználóbázisa milliós nagyságrendű, így rengeteg potenciális felhasználó számára nyílik lehetőség a játékba való kapcsolódáshoz. 
	Számos ilyen szolgáltatással találkozhatunk, elvárás volt a játékkal szemben, hogy minél több integrálható legyen, és a legismertebbek közül legalább kettő meg is legyen valósítva.	
	\item [Jutalmak] 
	A felhasználó számára elvégzett sportteljesítményei alapján jutalmakat kell kapnia, melyeket a játék közben felhasználhat. 
	\item [Érdeklődés fenntartása] 
	Olyan játékmenetet kell a kialakítani, amely a hosszabb távon is lekötik a játékos figyelmét. 
	Ezt elérendő fokozatosan nehezedő területeket kell a felhasználó számára kínálni, mely ösztönzi a továbbhaladásra.
	\item [Kis erőforrás igény] 
	A játéknak alacsony erőforrás mellett is megfelelően kell működnie, hogy az esetleges régebbi készülékeken is kielégítő játékélményt nyújtson. 
\end{description}


%3.1.1
\section*{Funkcionális követelmények}
\label{funkckovetelmenyeks}

Az alábbiakban a főbb funkcionális követelmények kerülnek bemutatásra. 

\begin{itemize}
	\item 
	A minél nagyobb számú támogatottság elérése érdekében az úgy kell kialakítani az alkalmazást, hogy a későbbiekben könnyedén lehessen integrálni különböző sport-tracker alkalmazást. 
	\item 
	Csatlakozás után az alkalmazásnak le kell töltenie a felhasználó legújabb sport tevékenységeit. 
	Erőforrás takarékosság szempontjából először meg kell bizonyosodni, hogy van-e új tevékenység. 
	Törekedni kell, hogy a felhasználóhoz tartozó összes adatot csak az első csatlakozás alkalmával, vagy más eszközön való bejelentkezés esetén töltsük le. 
	\item
	A különböző integrált alkalmazások adatainak tárolására létre kell hozni egy egységes adatszerkezetet, így elkerülve az inkonzisztens adatokat. 
	\item 
	Bejelentkezés után az újonnan letöltött adatok alapján a felhasználó staminát (kitartást) kap. 
	Egy játékosnak maximum 100 staminája lehet. 
	A kapott stamina mennyisége összhangban kell lennie ezzel a maximális értékkel, a játékos szintjével, és a tevékenységben szereplő adatok nagyságával. 
	Azaz az alacsony és magas szintű felhasználóknak is egyaránt élvezetesnek kell maradnia a játéknak, nem szabad se túl sokat, se túl keveset kapni. 
	Túl sok stamina esetén nagyon könnyen haladhatna a felhasználó a játékban, így egy idő után beleunna, túl kevés esetén viszont a folyamatosan túl nagy kihívást jelentő és csak nagy megerőltetést jelentő tevékenységek szintén ugyanezt a hatást érnék el.
	\item 
	A jutalomként megkapott staminát a felhasználó a játékosa fejlődésére használhatja fel különböző módokon. 
	Az egyik ilyen mód a világban való "barangolás", ami közben szörnyek támadhatnak a játékosra, amelyeket legyőzve játékbeli pénzt és tapasztalati pontot kap a játékos. 
	A másik mód küldetések vállalása, amelyet a felhasználónak kell ténylegesen sportolva teljesíteni, és csak a teljesítése után kapja meg az érte járó játékbeli jutalmat.
	\item 
	A játékos ezen kívül rendelkeznie kell tulajdonságokkal is, melyek a szörnyek elleni csatában segíthetnek számára. 
	Tulajdonságot növeli szintlépéssel vagy valamilyen kirívó sportteljesítményért cserébe lenne érdemes megengedni.
	\item 
	További tárgyakat is érdemes lenne megvalósítani a játékos számára, melyek védelemmel vagy támadóerővel növelhetnék a játékos erejét.
\end{itemize}


\section{Felhasznált technológiák}
\label{felhtechnologia}

\section*{Játékmotorok}
\label{jatekmotor}

Játékmotornak nevezzük a játékok - legyen az akár számítógépre vagy konzolra készült – azon részét, amely a program alapjául szolgáló technológiát adja. 
Szerepe, hogy megkönnyítse a fejlesztést illetve segítségével több platformon is futtatható lesz a játék.
Egy játék elkészítése az alapoktól nagyon nehéz, erőforrás-igényes feladat. 
Hamar világossá vált hogy szükség van olyan eszközökre, amelyek támogatják egy játék alapvető funkcióinak gyors implementálását, mint a megjelenítés és felhasználó input kezelése, hiszen ezek a legtöbb játék esetében nagy hasonlóságot mutatnak. 
Ezen funkciók megvalósítása után történhet az elkészülendő játék sajátosságainak kialakítása.
 
A fejlesztés megkezdése előtt több fajta játékmotort is megvizsgáltam abból a célból, hogy kiválasszam a legmegfelelőbbet a diplomamunkám elkészítéséhez. 

A fő szempontom az volt, hogy ingyenesen elérhető legyen, illetve illeszkedjen a választott játéktípus játékmenetéhez.

A két legnépszerűbb motorral kezdtem az ismerkedést, a Unity és az Unreal engine-ekkel. 
Mivel a programomat Android platformon terveztem elkészíteni, amit Java nyelven kell implementálni, ezek a motorok pedig a C++ nyelvet támogatják, így nem lehet közvetlenül Java nyelven használni őket ezért hamar kiestek. 
Méretük alapján túl nagynak is bizonyultak volna egy ilyen kisebb méretű projekthez. 
A következő játékmotor, amit megvizsgáltam a HexEngine volt, amit Szabó László készített el MSc diplomamunkájaként. 
A motor előnye, hogy rengeteg hasznos funkciót támogat szerepjátékok elkészítéséhez, viszont a játéktér hatszögű blokkokra van osztva, amivel megbonyolította volna a közlekedést a játékon belül.

A választásom így Hollósi Tamás által készített dream-iso-droid játékmotorra esett, amit témavezetőm ismertetett meg velem. 
Mivel készítője elérhető közelségben volt, ezért könnyebben sikerült megismerkednem a motor nyújtotta funkciókkal.

\Picture{Játékmotorok összehasonlítása}{3/gameengine}{width=10cm}

A dream-iso-droid egy olyan speciális játékmotor, ami kifejezetten Android platformra készült és a két dimenziós izometrikus nézetet támogatja. 
Ez a két funkciója pontosan megfelelt az elvárásaimnak, amit a játékmotor felé támasztottam, aminek segítségével fejleszteni szerettem volna az RPG játékomat.
A \ref{fig_3/gameengine}. ábrán látható egy összefoglaló táblázat, melyben megtekinthető a megvizsgált játékmotorok azon tulajdonságai, melyeket figyelembe vettem a választás során. 

\section*{Fejlesztőkörnyezet}
\label{ide}

A megvalósítás során az Android Studio fejlesztőkörnyezetet használtam, melyet az Android operációs rendszer fejlesztője, a Google készített el. 
Integrálva van benne minden olyan szolgáltatás mely nagyban megkönnyíti a fejlesztők munkáját, mint például a Gradle keretrendszer amely többek közt a projekt építéséért és a különböző külső függőségekért felelős. 
Továbbiakban található benne grafikus felületszerkesztő is, ahol drop\&down módszer segítségével a grafikus felület szerkezetét könnyen össze tudjuk rakni. 
Integrálva van tovább több verziókövető is, így a fejlesztőkörnyezet elhagyása nélkül tudjuk az újabb verziójú fájlokat a megfelelő távoli tárolóoldalra eljuttatni. 

\section*{OAuth protokoll}
\label{oauth}

\Picture{Az OAuth protokoll absztrakt működési ábrája}{3/oauthflow}{width=14cm}

Az OAuth protokoll \cite{oauthprotocol} egy nyílt autorizációs szabvány mely segítségével a felhasználók megoszthatják bizonyos privát információit anélkül, hogy azonosítási adataikat kiadnák. 
A \ref{fig_3/oauthflow}. ábrán látható a protokoll működési folyamata. 
Ha egy alkalmazás el akar érni olyan adatot ami bizalmasan van kezelve, ahhoz előbb engedélyt kell kérnie hozzá. 
Az alkalmazás továbbirányítja a felhasználót az adott szolgáltatás felületére - többnyire valamilyen webböngészőbe -, ahol engedélyt tud adni számára. 
Ekkor a felhasználónak be kell jelentkeznie a fiókjába, és megadni a kért engedélyeket. 
Az engedély megadása utána a hitelesítő szerver egy megerősítő kódot juttat el az alkalmazás számára. 
Az alkalmazás ezt a megerősítő kódot tudja "elcserélni" a szerverrel egy tokenre. 
A későbbi adatelérés alkalmával minden kéréshez csatolnia kell az alkalmazásnak ezt a tokent. 
A szerver ezt a tokent vizsgálva tudja eldönteni, hogy a kért információhoz a felhasználó engedélyt adott-e. 

\section*{OrmLite}
\label{ormlite}

Az alkalmazáson belül az adatok hosszútávú tárolására az OrmLite könyvtárat \cite{ormlite} használtam, amely képes Java objektumokat az alkalmazás helyi SQLite adatbázisába írni. 
Több platformon is elérhető, az Androidos készülékeken natív API hívásokkal kezeli az adatbázist. 
Használata könnyű, egyszerű Java osztályokat kell a megfelelő annotációkkal ellátni, majd az annotációk alapján a könyvtár képes a kívánt módon kiolvasni avagy eltárolni az adatokat. 
A könyvtár a DAO, azaz Data Access Object tervezési mintát használja, amely elkülöníti az alacsony szintű API hívásokat a magasabb szintű szolgáltatásoktól. 


\chapter{Fejlesztői dokumentáció}
\label{devmanual}
\Aref{fig_5/architektura}. ábrán látható a játék architektúrája. 
Két fő részre bontható fel, egy megjelenítő egységre és egy logikai egységre. 
A megjelenítő egység elkészítéséhez Hollósi Tamás által készített dream-iso-droid nevű keretrendszert használtam fel. 
A játékban megjelenő grafikai elemek túlnyomó részét ez kezeli. 
A logikai egység feladata a megtervezett szabályok és játékmechanikai elemek felügyelete. 
Ez az egység több kisebb modulra bontható fel, melyeknek mind megvan a saját jól elkülöníthető feladata. 
Az egység ezeket a modulokat kezeli, és a belső működésükbe nem szól bele, elfedve azokat. 
Elkerülhetetlen, hogy két modul kommunikáljon egymással, ezt is a logikai egység szabályozza. 
A két egység szoros együttműködéseként valósul meg a játék. 
A továbbiakban bemutatásra kerülnek a megvalósítás részletei. 

\Picture{A program architektúrája}{5/architektura}{width=14cm}

\subsection*{Engedélyek, rendszerkövetelmények}
\label{requirements}

Ahhoz, hogy a játék minden funkcionalitása elérhetővé váljon néhány rendszerkövetelménynek meg kell felelnie a felhasználó eszközének. 
Az eszközön legalább 4.4-es Android operációs rendszernek kell futnia, ami a 19-es API szintnek felel meg.
Ezt feltételt a \Highlight{build.gradle} fájlban adhatjuk meg, ahol más, fontos paramétereket is megadunk. 
Ebben a fájlban kell megadni többek közt, hogy mely harmadik féltől származó könyvtárakat használ az alkalmazás, és ezen könyvtárak használt verziószámát is. 

Az alkalmazásnak csupán egy engeélyre van szüksége, amit az \Highlight{AndroidManifest.xml} fájlban határozhatunk meg a következő módon:

\begin{lstlisting}
    <uses-permission android:name="android.permission.INTERNET" />
\end{lstlisting}

Ezen kívül a fájlban az alkalmazást leíró metaadatok, általános konfigurációs beállítások találhatóak, például hogy melyik legyen az induló Activity.

\section{A játék indítása}
\label{jatekinditas}

Az alkalmazás indítása után a felhasználót egy egyszerű menü fogadja a \Highlight{StartActivity} nevű osztályban. 
A menü tetején található a játék logója, melyet a játékban található több grafikai elemmel együtt Böröndi Evelin hallgatótársam készített el számomra. 
Itt két lehetőség tárul a felhasználó elé, képes csatlakozni új sport-nyomkövető szolgáltatásokhoz, vagy elkezdhet játszani. 
A csatlakoztatható szolgáltatások egy egyszerű listában tárolódnak soronként, ami tartalmazza a nevét, és egy gombot, amely elindítja a csatlakozási folyamatot. 

\subsection*{Csatlakozás nyomkövető alkalmazáshoz}
\label{trackerconnect}
Jelenleg a két legnagyobb felhasználói bázissal rendelkező sport-trackerek a Runkeeper és a Strava. 
Mindkettő rendelkezik publikusan elérhető API-val \cite{runkeeperapi} \cite{stravaapi}, mely segítségével könnyedén készíthetünk hozzájuk saját alkalmazásokat. 
Mindkét szolgáltatás esetén ahhoz, hogy a játékkal össze tudja kötni a felhasználó a sport-tracker fiókját az OAuth protokollra van szükségünk, melyet a \ref{oauth} fejezeteben mutattam be. 

A protokoll teljes működésének implementálása sok biztonsági és hibakezelési kérdést vet fel, így úgy döntöttem, hogy egy már kész könyvtárat használok hozzá. 

A választott könyvtáram a ScribeJava \cite{scribejava}, mely a protokoll több verzióját is támogatja és számos szolgáltatáshoz már kész API-val rendelkezik. 
Az általam integrált két sport-nyomkövető nem voltak elkészítve, így ezeket nekem kellett megvalósítanom. 
Szerencsére a könyvtár úgy lett kialakítva, hogy minden, a protokollt használó szolgáltatáshoz egységesen lehessen API-t készíteni. 

A Runkeeper nyomkövetőhöz elkészített API legfontosabb része a következőként néz ki:

\begin{lstlisting}
    private static final String AUTHORIZATION_URL = "https://runkeeper.com/apps/authorize?client_id=%s&response_type=code&redirect_uri=%s";
    private static final String ACCESS_TOKEN_URL = "https://runkeeper.com/apps/token";

    @Override
    public String getAccessTokenEndpoint() { return ACCESS_TOKEN_URL; }

    @Override
    public String getAuthorizationUrl(OAuthConfig config) {
        Preconditions.checkValidUrl(config.getCallback(), "Must provide a valid url as callback.");
        final StringBuilder sb = new StringBuilder(String.format(AUTHORIZATION_URL, config.getApiKey(), OAuthEncoder.encode(config.getCallback())));
        // ...
        return sb.toString();
    }
\end{lstlisting}

Mivel ez a kommunikáció hálózati tevékenységgel jár, nem történhet az Android fő programszálán. 
Egyrészt ha a fő szálon folyna ez a kommunikáció, az alkalmazás nem tudna tovább futni, amíg hitelesítési folyamat be nem fejeződik. 
Ez akár több másodpercbe is telhet a hálózati körülményeket figyelembe véve, így addig az alkalmazás blokkolódna. 
A felhasználói élmény miatt ez nem megengedhető, így ezt a folyamatot a háttérben kell elvégezni, hogy az alkalmazás zavartalanul futhasson tovább. 
Az Android SDK-ban több beépített lehetőség segítségével is meg tudjuk valósítani ezt:

\begin{itemize}
	\item Egyszerű Java szálak 
	\item AsyncTask 
	\item IntentService 
\end{itemize}

Ezeknek a lehetőségeknek megvan a maguk előnye és hátránya. 
A normál Java szálak használata széles körben elterjedt, de nagyobb, bonyolultabb programszerkezet mellett használatuk nehézkes. 
A következő választási lehetőség az Android SDk-ban bemutatott AsyncTask osztály használata. 
Ennek segítségével könnyedén indíthatunk háttérben futó kódrészleteket. 
Az osztályon belül felülírható metódusok, amelyek az adott tevékenység elején, közben, vagy a feladata végeztével hívódnak meg. 
Ez a fajta megoldás sokkal kötetlenebb, viszont ahogy az előző pontban taglalt sima szálhoz hasonlóan, ha az adott Activity, amelyikből el lett indítva háttérbe kerül, az operációs rendszer meg tudja szakítani a folyamat futását. 
Az \Highlight{IntentService} esetén ez nem történik meg, tovább egyrészt nincs \Highlight{Activity}-hez kötve, másrészt a kéréseket egy sorba teszi, amelynek a későbbieknek még fontos szerepe lesz. 

A háttérben folyó munkákért a \Highlight{FetchService} nevű osztály felelős, mely a beépített \Highlight{IntentService} osztályból származik, és felülírja az \Highlight{onHandleIntent()} metódust. 
Ez a metódus már a háttérben fut. Paraméterként egy Intent-et kap, melyet a \Highlight{FetchService} osztály statikus metódusain keresztül hozunk létre, így példányosítás nélkül is képesek vagyunk ilyen folyamatot indítani. 
Egy adott sport-trackerhez tartozó összes eddigi elmentett sportteljesítmény letöltéséhez szükséges \Highlight{Intent} elkészítése az alábbi kódrészletben látszódik. 

\begin{lstlisting}

public static void startFetchActivities(Context context, TrackerService tracker) {
	Intent intent = new Intent(context, FetchService.class);
	intent.setAction(ACTION_FETCH_ALL_ACTIVITY);
	intent.putExtra(EXTRA_TRACKER_SERVICE, tracker);
	context.startService(intent);
}

\end{lstlisting}

Ez a metódus egy sport-tracker alkalmazáshoz való sikeres csatlakozás esetén automatikus meghívódik. 
A beállított akció alapján tudjuk majd meghatározni az \Highlight{onHandleIntent()} metódusban, hogy milyen típusú műveletet kell végrehajtani. 

Mivel az itt futó kód futási ideje nemdeterminisztikus, előre nem tudhatjuk hogy mikor ér véget, hisz nagyban függ az elérhető hálózati csatlakozások jelerősségétől, így nem tudjuk mikor engedhetjük tovább a játékost a játékfelületre. 
Ahhoz, hogy valahogy tudjuk mikor ért véget az adatok letöltése, az \Highlight{IntentService} esetén rendelkezésre áll a \Highlight{BroadcastReceiver} osztály. 
Amint az összes letöltés befejeződött ezt el kell juttatnunk az \Highlight{Activity} számára, ahol a gombot megnyomtuk. 
Az üzenet létrehozása során egy új \Highlight{Intent}-et hozunk létre speciális akció típussal, amelyet broadcast üzenet formájában kiküldünk. 
Az ilyen broadcast üzenetekre feliratkozhatnak a különböző activity-k, további az is lehetséges, hogy csak bizonyos akciójú broadcast üzenetekre reagáljon, mely implementálásának főbb részlete az alábbi kódrészletben figyelhető meg:

\begin{lstlisting}
        IntentFilter mStatusIntentFilter = new IntentFilter();
        mStatusIntentFilter.addAction(FETCH_NEW_ACTIVITY_DONE);
        mStatusIntentFilter.addAction(FETCH_ALL_ACTIVITY_DONE);
        LocalBroadcastManager.getInstance(this).registerReceiver(receiver, mStatusIntentFilter);
\end{lstlisting}

Ebben az esetben regisztrálunk egy fogadó osztályt, melyet a \Highlight{BroadcastReceiver} örököltettünk és \Highlight{onReceive} metódusát felülírtuk. 
A második paramétere az az \Highlight{Intent} lesz, amelyet a háttérben futó kód befejeztével sugároztunk szét, és csak is akkor jutunk el ide, ha az adott intent objektum akciója megfelel a szűrőbe beállított akciók valamelyikének. 
Amennyiben nem csatlakozni akarunk egy lehetséges sport-trackerhez, hanem el szeretnénk kezdeni játszani, abban az esetben hasonló folyamat játszódik le, annyi eltéréssel, hogy a már csatlakoztatott trackerektől csakis a legújabb sportteljesítményeket kérjük le. 








%% <== End of hints
%%%%%%%%%%%%%%%%%%%%%%%%%%%%%%%%%%%%%%%%%%%%%%%%%%%%%%%%%%%%%


%%%%%%%%%%%%%%%%%%%%%%%%%%%%%%%%%%%%%%%%%%%%%%%%%%%%%%%%%%%%%
%% BIBLIOGRAPHY AND OTHER LISTS
%%%%%%%%%%%%%%%%%%%%%%%%%%%%%%%%%%%%%%%%%%%%%%%%%%%%%%%%%%%%%
%% A small distance to the other stuff in the table of contents (toc)
\addtocontents{toc}{\protect\vspace*{\baselineskip}}


\bibliography{cite}
\bibliographystyle{mybibstyle}

\newpage

\Large
\begin{center}
	\textbf{MELLÉKLET}
\end{center}
\normalsize
\noindent
A mellékelt CD könyvtárszerkezete


% \begin{itemize}
%    \item \textbf{Dokumentum}
%    \begin{itemize}
%        \item \textbf{Forrás} - A szakdolgozat szerkeszthető formátumban
%        \item \textbf{Hivatkozások} - A szakdolgozatban lévő internetes hivatkozások letöltve
%        \item szakdolgozat.pdf
%    \end{itemize}
%    \item \textbf{Forrás} - A program forrásállománya
%    \begin{itemize}
%        \item \textbf{HexEngine}
%        \item \textbf{HexEngine-android}
%        \item \textbf{HexEngine-desktop}
%    \end{itemize}
%    \item \textbf{Program} - A futtatható program
%    \begin{itemize}
%        \item \textbf{bin} - A program grafikai és konfigurációs fájljai ami szükségesek az indításhoz
%        \item \textbf{hav} - A program a mentéseket tárolja itt
%        \item \textbf{java\_telepito} - A Java környezet telepítői
%        \item \textbf{libs} - Futtatáshoz kapcsolódó java fájlok
%        \item starter.jar
%    \end{itemize}
% \end{itemize}





%%%%%%%%%%%%%%%%%%%%%%%%%%%%%%%%%%%%%%%%%%%%%%%%%%%%%%%%%%%%%
%% APPENDICES
%%%%%%%%%%%%%%%%%%%%%%%%%%%%%%%%%%%%%%%%%%%%%%%%%%%%%%%%%%%%%
\appendix
%% ==> Write your text here or include other files.

%\input{FileName} %You need a file 'FileName.tex' for this.


\end{document}

